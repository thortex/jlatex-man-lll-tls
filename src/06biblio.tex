%#!platex tls.tex
\chapter{文献管理}

\begin{abstract}
 
\end{abstract}

\section{\BibTeX の簡単な使い方}

\BibTeX とは\Z{文献管理}用プログラムです。実際は日本語化された \JBibTeX
を使うことになると思います。まずは具体例を見た方が早いでしょう。 2000年
発行の渡辺徹による『これが私の生きる道』という偽物出版から出ている本を、
参考文献として参照しているとしましょう。これを \JBibTeX の書式通りに記述
すると \Va{file}{bib}に,
\begin{InText}
@book{watanabe2000a,
  author    = {渡辺 徹},
  yomi      = {Toru Watanabe},
  title     = {これが私の生きる道},
  year      = {2000},
  publisher = {偽物出版}
 }
\end{InText}
というファイルがあるとします。これを \Va{file}{tex} で

\begin{InText}
\documentclass{jarticle}
\bibliographystyle{jplain}
\begin{document}
\nocite*
\bibliography{file}
\end{document}
\end{InText}

としてターミナルから
\begin{InTerm}
 \type{platex } \va{file}
 \type{jbibtex } \va{file}
 \type{platex } \va{file}
 \type{platex } \va{file}
\end{InTerm}

とすれば、 \Va{file}{bbl}, \Va{file}{blg}, \Va{file}{dvi} というファイル
が作成されるので,内容を確認してみて下さい。

\begin{description}
 \item[\Va{file}{bbl}]
    \Va{file}{bib} から \JBibTeX が作成した \TeX 用 文献一覧
 \item[\Va{file}{blg}]
    \JBibTeX のログファイル
 \item[\Va{file}{dvi}]
    \TeX の出力ファイル
\end{description}

\Va{file}{bbl} の中身を確認すると
\begin{InText}
 \begin{thebibliography}{1}
 \bibitem{watanabe2000a} 
 渡辺徹. \newblock これが私の生きる道. \newblock 偽物出版, 2000.
 \end{thebibliography}
\end{InText}

となっております。これを \prog{xdvi} などで \Va{file}{dvi} を確認すると

\begin{OutText}
 {\large\gtfamily 参考文献}\\ \relax
 [1] 渡辺徹. これが私の生きる道. 偽物出版, 2000.
\end{OutText}

となっております。


\section{Emacs \BibTeX モード}

\begin{append}
 ここにEmacs \BibTeX モードの使い方について解説する.
\end{append}

\section{JabRef}

旧 J\BibTeX Manager の後継で文献管理プログラム。 \Prog{Java} 
で動作するため、プラットフォームに依存することはない。 
公式サイトから本体をダウンロードする事が出来る。インストールに
関しては本家サイトに書いているとおりだが、 Java SDK 1.4.2 
でも導入すれば良いだろう。 JabRef 自体は Sourcefouge から
ダウンロードできる。実行方法は
\begin{InTerm}
  \type{java -jar JabRef-1.7b.jar &}
\end{InTerm}

などとすれば良いだけ。Windows の場合も
\begin{InTerm}
 \type[]{start javaw -jar JabRef-1.7b.jar}
\end{InTerm}
とかいうバッチファイル \fl{JabRef.bat} を \Fl{JabRef-1.7b.jar} があるフォ
ルダにおくだけ。

文字化けするときは
\begin{InTerm}
 \type{java -Dfile.encoding="SJIS" -jar JabRef.jar}
 \type{java -Dfile.encoding="EUC_JP" -jar JabRef.jar}
\end{InTerm}
として文字コードを指定してください。


%\section{ama2bib}




\section{\texorpdfstring{\BibTeX2HTML}{BibTeX2HTML}}

というファイル \Va{file}{bib} が存在し

 bibtex2html file.bib

とすれば \Va{file}{html}, \va{file}\fl{\_bib.html} の二つのファイルが生成され
る。 \Va{file}{html} 実際の文献一覧のファイル、 \va{file}fl{\_bib.html} は生
の \BibTeX 形式のファイル \va{file}{bib} を HTML で閲覧できるようにした
もの。

\section{\BibTeX エントリ一覧}


\begin{append}
 ここにフィールドやエントリの一覧を追加する.表にまとめたり,
分かりやすいように工夫する.
\end{append}

