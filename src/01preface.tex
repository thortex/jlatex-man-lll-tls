\chapter{まえがき}

\section*{任意の角度の直線が引きたい}

\TeX は文字などの\Z{箱}を\Z{ユークリッド座標系}に配置するだけのソフトで,
図を表示したりグラフを描画する機能は持っていません.あくまで,
ある座標に「高さ」「幅」「深さ」「\Z{リガチャ}」「\Z{ペアカーニング}」な
どの\Z{メトリクス情報}を持った箱を上から下へ,左から右へ置いて行く高貴な
タイプセッタなのです.

そもそも任意の角度の直線が引きたいと思っても, 
$\theta = \{0,\pi/6,\pi/4,\pi/3,\pi/2\}$ 以外の直線は\Z{ラスタライジング}
がデバイスによって異なります.そのため必然的に\Z{任意の角度の直線}を引くとい
う現実世界では当たり前のように行われている\Z{幾何操作}でもコンピュータの
世界で実装するのは大変だったりします.

そこで \TeX の力ではなく,\Z{周辺ツール}の力を借りて図やグラフを
描画する事が必要となります.本冊子ではそれらのプログラムの大雑把な
使用例と \TeX との連携方法を紹介します.周辺ツールを使いこなすための
過剰な解説はこの冊子ではしません.巷にはそれらフリーのツールの
マニュアルがごろごろありますので,それらを参照してください.
%後付けに参考資料として URL などを記述します.

さらに \TeX は\Z{プレインテキスト} (\Z{plain text}) で原稿を記述しますか
ら,原稿を執筆するためのテキストエディッタやそれに類似する統合環境が
必要となります.古典的な環境でいえば \Prog{Emacs}, \prog{Ya\TeX},
\Prog{xdvi} などでしょう.

この冊子ではそれら,\TeX を取り巻く周辺のツールの簡単な使い方と,
\TeX との連携方法を紹介します.余り深い部分まで立ち入りませんので,
より細かく詳しい使い方が知りたくなったのならば,それらツールに付属する
完全なマニュアルを参照してください.

\section*{この冊子の執筆方針}

このような冊子の作成においてもそれは同様で,一人で作る場合と集団で執筆す
るのとでは方向性や手法が異なります.そこで,私はこの冊子を基本的に私の独
断で編集することにしています.読者の方からの意見や要望,質問等は大歓迎で
すが,市場の需要や経済性等はほとんど考慮していません.要するに趣味で提供
している代物なので,明後日の方向に進んでいます.しかし,現在書店に並んで
いる \LaTeX 絡みのハウツー本 (適切な表現とは言いがたいが) よりはクオリティ
の高いものを提供しようと考えています (『だれでもできる○○』とか,そうい
う本を良く見掛けますが,\LaTeX はそんなに浅いプログラムではないと思って
います).\laTEX は噛めば噛む程その味が分かってきます.是非とも長旅の準備
をして,この大航海につき合っていただきたいものです.

%\section*{}
%
%私が趣味で執筆している冊子なので,明後日の方向に進んでいるのは勘弁してい
%ただいて,一応この冊子もコンパスをもって航海することにしています.その
%方向は『初心者が書籍作成やレポート/論文作成の段階でおそらく必要になるだろ
%うマクロパッケージの抜粋』にあります.

%\clearpage
\section*{凡例}

本文において\Z{キーボード}の特定の\emph{キートップ}を示すときに
は \key{X} と表します。複数のキーを同時に押すことを \key{Ctrl,Alt,F1} と
します。あるキー \key{X} の後に続けて他のキー \key{Y} を入力すること
を \key{X}\key{Y} と表記します。キーボードの入力で\emph{スペース}を示す
時は`\vs'とし、\Z{エンターキー}や\Z{リターンキー}を示す時
は`\return'とします。

\Z{コンソール}からの入力で先頭に\Z{パーセント} `\%' がある場合は
\emph{一般ユーザ権限}での実行を、\Z{シャープ} `\#' がある場合は
\emph{管理者権限} (root など) での実行を示し、パーセントや
シャープの入力を促すものではありません。

\va{コマンド名} として\Z{角括弧}で括られたものは\Z{変数}を意味します。
実際には何らかの文字列、例えば`\cmd{newcommand}'等に置き換わります。
「コマンド名」そのものを示しているわけではありません。

\par\vskip.5zw

%文中で \raise1zw\hbox{\reversedvideodbend} が先頭にある段落は、その操作
%に\emph{注意}する事を促しています。\manstar が先頭にある場合は冗長な手順
%を省略するための\emph{ヒント}が書かれています。 \manerrarrow の時は危険
%である事を\emph{警告}します。

\emph{ASCII} コード \texttt{0x5C} の\emph{バックスラッシュ} `\bs' は、
\Prog{Linux} や \Prog{Mac OS X} ではその通りに表示されますが、
Windows ではほとんどの場合\emph{円記号} `\yen' として表示されます。
意味的には同じですので、誤解されないように注意してください。

%本文書に記載されている企業・団体の名前や製品名等はそれぞれの権利帰属者
%の\Z{商標}または\Z{商標登録}であり所有物です。本冊子で
%は`\texttrademark'及び`\textregistered'は明記していません。


本冊子では書体を変更することによって同じ語句でも
違った意味を持つものが多数あります.\qu{\prog{dvips}}
という語があったとしても\qu{\textsf{dvips}}や
\qu{\texttt{dvips}},\qu{\textsl{dvips}},\qu{\textit{dvips}}
はすべて別の意味を持っています.
%これらの書体の種類については\secref{sec:font}を参照してください.
\begin{center}
 \begin{tabular}{lll}
 \hline
 書体          & 意味      & 例\\
 \hline
 ローマン体    & 通常の文章& \textrm{dvips}\\
 サンセリフ体  & パッケージやクラス& \textsf{dvips}\\
 タイプライタ体& キーボードからの入力など& \texttt{dvips}\\
 イタリック体  & 変数や強調& \textit{dvips}\\
 スラント体    & オプション& \textsl{dvips}\\
 \hline
 \end{tabular}
\end{center}





