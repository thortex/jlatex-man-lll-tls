% \iffalse meta-comment
% Copyright (C) 2004 by Thor Watanabe
% 
% This file may be distributed and/or modified under the
% conditions of the LaTeX Project Public License, either
% version 1.2 of this license or (at your option) any later
% version.  The latest version of this license is in:
% 
%    http://www.latex-project.org/lppl.txt
% 
% and version 1.2 or later is part of all distributions of
% LaTeX version 1999/12/01 or later.
% 
% \fi
% \CheckSum{0}
% \iffalse
%<lll>\NeedsTeXFormat{LaTeX2e}
%<lll>\ProvidesPackage{lll}
%<lll>                        [20005/05/29 v0.1a Thor]
%<*driver>
\documentclass{jsarticle}
\usepackage{doc}
\usepackage{shortvrb}
\newcommand{\marg}[1]{{\mbox{$\langle$}\string #1\mbox{$\rangle$}}}
\MakeShortVerb|
\EnableCrossrefs
\CodelineIndex
\DoNotIndex{\textsf,\bs,\texttt,\glossary,\index,\zdash}
\DoNotIndex{\def,\DescribeMacro,\@ifnextchar,\mbox,\newcommand}
\DoNotIndex{\hspace,\normalfont,\newfont,\string,\protect}
\DoNotIndex{\@ne,\@for,\@tempcnta,\@tempcntb,\addvspace}
\DoNotIndex{\advance,\al,\begin,\begingroup,\bfseries}
\DoNotIndex{\char,\color,\colorbox}
\DoNotIndex{\DeclareRobustCommand}
\DoNotIndex{\endlist,\varbatim,\endvarbatim}
\DoNotIndex{\fullwidth,\hbox,\hskip,\immediate}
\DoNotIndex{\itshape,\kern,\member,\newenvironment}
\DoNotIndex{\parskip,\par,\parindent,\put,\ref,\relax}
\DoNotIndex{\rightmargin,\rmfamily,\small,\TeX,\textcolor}
\DoNotIndex{\textsl,\ttfamily,\upshape,\usepackage}
\DoNotIndex{vector,\vskip}
\DoNotIndex{\end}
\DoNotIndex{\endcsname}
\DoNotIndex{\hfill}
\DoNotIndex{\hfil}
\DoNotIndex{\hss}
\DoNotIndex{\ifhum}
\DoNotIndex{\let}
\DoNotIndex{\RequirePackage}
\DoNotIndex{\z@}
%
\addtolength{\textwidth}{-1in}
\addtolength{\evensidemargin}{1in}
\addtolength{\oddsidemargin}{1in}
\addtolength{\marginparwidth}{1in}
\setlength\marginparsep{10pt}
\setlength\marginparpush{0pt}
\setcounter{StandardModuleDepth}{1}
\setcounter{tocdepth}{2}
\setcounter{secnumdepth}{1}
%
\GetFileInfo{lll.sty}
\begin{document}
  \DocInput{lll.dtx}
\end{document}
%</driver>
% \fi
%
% \title{『好き好き{\LaTeXe} シリーズ』用マクロ集}
% \author{渡辺徹}
% \date{2005/03/17}
% \maketitle
% \tableofcontents
% \StopEventually{}
%    \begin{macrocode}
%<*lll>
%    \end{macrocode}
%
%
% \subsection{表題情報}
%
% \DescribeMacro{\jFileDate}
% \DescribeMacro{\jFileVersion}
% 冊子の配布日とバージョン情報を保持する|\FileDatej|と
% |\FileVerjou|を定義します.
%    \begin{macrocode}
% filename: lll.sty
% author:   Thor Watanabe
% email:    thor(at)tex(dot)dante(dot)jp
% date:     2005/05/29
% version:  0.1a
% copying:  lppl
% webpage:  http://tex.dante.jp/typo/
%
%\global\let\@jFileVersion\@empty
%\def\jFileVersion#1{\gdef\@jFileVersion{#1}}
\def\jFileDate{\number\year/\number\month/\number\day}
%
\def\jFileVersion{0.1}
%    \end{macrocode}
%
%
% \DescribeMacro{\author}
% \DescribeMacro{\title}
% \DescribeMacro{\date}
% |\author|と|\title|と|\date|は固定です.
%    \begin{macrocode}
\author{渡辺徹}
\title{好き好き{\LaTeXe} 周辺ツール編}
\date{\today}
%    \end{macrocode}
%
%
% \DescribeMacro{\contact}
% \DescribeMacro{\@contact}
% 冊子の作成者への連絡先を定義します.これは
% 表紙に使われます.
%    \begin{macrocode}
% \contact{Author's Address}, \FileVerj{<verison>}
% 後記のtitlepage環境に必要
\global \let \@contact \@empty
\def \contact#1{\gdef\@contact{#1}}
%
\contact{%
   \parbox{.9\textwidth}{Th\'or Watanabe\\%
   Dept.\ of System Information Science\\%
   Future University-Hakodate\\%
   thor(at)tex(dot)dante(dot)jp\\%
   \url{http://tex.dante.jp/}}}%
%    \end{macrocode}
%
% 
% \subsection{必要となるパッケージ群}
%
% \DescribeMacro{\RequirePackage}
% 『好き好き』が必要とするマクロたちです.
% マクロを読み込むときに|\ifusedvipdfm|等のブール値を
% 使ってパッケージオプションを判定します.これは主原稿
% で定義されています.\textsf{listings}は様々な場面で
% 使われます.\textsf{jlisting}は通称しっぽ愛好家氏より
% 頂いたものです.\textsf{okumacro}からは|\ruby|,
% |\yen|,|\key|,|\return|,|\MARU|,
% |\JTeX|,|\JLaTeX|,|screen|環境などを
% 借用しています.箇条書き用の|namelist|環境や
% 文献一覧を出力する|mybibliography|環境も使っています.
% \textsf{url}パッケージからは|\url|命令などを使っています.
% 
%    \begin{macrocode}
%\RequirePackage{listings}%http://www.atscire.de/
%\RequirePackage{jlisting}%http://www.geocities.jp/unzoo_sa/
%\RequirePackage{okumacro}%http://www.matsusaka-u.ac.jp/~okumura/
%\RequirePackage{emathN}% http://emath.s40.xrea.com/

\RequirePackage[usenames]{color}[1999/02/16]%
\RequirePackage{graphicx}[1999/02/16]%
\RequirePackage{epic}\relax%
\RequirePackage{eepic}\relax%
\RequirePackage{pict2e}[2004/02/19]%
%
\RequirePackage{amsmath}[2000/07/18]
\RequirePackage{amssymb}[2002/01/22]
\RequirePackage{array}[2003/12/17]
\RequirePackage{bm}[2004/02/26]
\RequirePackage{booktabs}[2003/03/28]
\RequirePackage{calc}[1998/07/07]
\RequirePackage{cmtt}[1996/05/25]
\RequirePackage{delarray}[1994/03/14]
\RequirePackage{dcolumn}[2001/05/28]
%\RequirePackage{dropping}[1997/06/12]
\RequirePackage{enumerate}[1999/03/05]
\RequirePackage{fancybox}[2000/09/19]
\RequirePackage{ifthen}[2001/05/26]
\RequirePackage{labelfig}\relax
\RequirePackage{latexsym}[1998/08/17]
\RequirePackage{leftidx}\relax
\RequirePackage{listings}[2004/02/13]% v1.2
\RequirePackage{longtable}[2004/02/01]
\RequirePackage{makeidx}[2000/03/29]
\RequirePackage{manfnt}\relax
\RequirePackage{mflogo}[1999/03/10]
\RequirePackage{multicol}[2004/02/14]
\RequirePackage{multirow}\relax%
\RequirePackage{tabularx}[1999/01/07]
\RequirePackage{theorem}[1995/11/23]
\RequirePackage[obeyspaces,spaces]{url}\relax
\RequirePackage{verbatim}[2003/08/22]
\RequirePackage{wrapfig}[2003/01/31]
\RequirePackage{xspace}[1997/10/13]
%
\RequirePackage{jlisting}[2004/03/24]
%
%\ifx\pfmtname\@undefined\else
%   \IfFileExists{emathN.sty}{%
%      \RequirePackage {emathN}%
%      \AtBeginDocument{\resettagform}%
%   }{}%
%\fi
%
\ifx\pfmtname\@undefined\else
  \RequirePackage {okumacro}[2003/11/24]%
\fi
% The cite package breaks hyperref
%\RequirePackage {cite}\relax
% Font settings
\RequirePackage[T1]{fontenc}%
\RequirePackage{textcomp}%
% type1 font is better.
\IfFileExists{type1ec.sty}{%
   \RequirePackage{type1ec}%
}{%
   \IfFileExists{lmodern.sty}{%
      \RequirePackage{lmodern}%
   }{%
      \IfFileExists{txfonts.sty}{%
         \RequirePackage{txfonts}%
      }{%
         \IfFileExists{pxfonts.sty}{%
            \RequirePackage{pxfonts}%
         }{%
            \IfFileExists{type1cm.sty}{%
               \RequirePackage{type1cm}%
            }{%
               \PackageWarning{You may get type3 fonts in your\space
                  \jobname.pdf or \jobname.ps Please try to get\space
                  type1 font package(s)}%
            }%
         }%
      }%
   }%
}%
%
%    \end{macrocode}
%
% \DescribeMacro{\hogeHyper}
% \textsf{hyperref}を使っていた場合にはPDF文書情報やPDFしおり
% 等の設定が必要になりますので,|\hogeHyper|で判断しています.
%    \begin{macrocode}
\ifx\hogeHyper\@undefined\else
\newif\ifHyper
\Hypertrue
\RequirePackage[dvipdfm,%
   bookmarks=true,%
   bookmarkstype=toc,%
   bookmarksnumbered=false,%
   bookmarksopen=true,%
   colorlinks=true,%
   linkcolor=blue,%
   citecolor=blue,%
   filecolor=blue,%
   menucolor=magenta,%
   pagecolor=blue,%
   urlcolor=blue,%
   plainpages=false%
]{hyperref}
\special{pdf:docinfo <<
   /Author   ( Thor Watanabe )
   /Title    ( Love Love LaTeX2e )
   /Subject  ( For Useful tools related LaTeX)
   /Creator  ( pLaTeX2e with hyperref and dvipdfmx )
   /Keywords ( TeX, LaTeX, LaTeX2e, pTeX, pLaTeX, pLaTeX2e) >>}%
\AtBeginDocument{%
  \def\theHProb{\theHchapter.\arabic{Prob}}%
  \def\theHExe{\theHchapter.\arabic{Prob}}%
  \def\theHItem{\theHchapter.\arabic{Item}}%
}
\fi
%    \end{macrocode}
% 
% \subsection{半面パラメータなど} 
% 
% \DescribeMacro{\ftextwidth}
% \DescribeMacro{\ffullwidth}
% |\fboxsep| や |\fboxrule| の幅を引いた値を |\ftextwidth| などに
% 代入します。
% \DescribeMacro{\marginparpush}
% \DescribeMacro{\footskip}
% 
%
%    \begin{macrocode}
\addtolength{\topmargin}{-2.5\baselineskip}
\addtolength{\textheight}{3\baselineskip}
% \ftextwidth
\newlength{\ftextwidth}
\setlength{\ftextwidth}{\textwidth}
\addtolength{\ftextwidth}{-2\fboxsep}
\addtolength{\ftextwidth}{-2\fboxrule}
% \ffullwidth
\newlength{\ffullwidth}
\setlength{\ffullwidth}{\fullwidth}
\addtolength{\ffullwidth}{-2\fboxsep}
\addtolength{\ffullwidth}{-2\fboxrule}
% 
\setlength\marginparpush{.5\baselineskip}
\setlength\footskip{3zw}
%    \end{macrocode}
%
% \DescribeMacro{\headfont}
% 見出し用のフォントスタイルを指定する |\headfont| を再定義します。
%
%    \begin{macrocode}
\ifx\headfont\@undefined\else
  \renewcommand\headfont{\normalfont\gtfamily\sffamily}%\bfseries}%
\fi
%    \end{macrocode}%
%
% \subsection{\textsf{listings}/\textsf{wrapfig}用の設定}
%
% \DescribeMacro{\lstset}
% \DescribeMacro{\lstlistlistingname}
% \DescribeMacro{\lstlistingname}
% \textsf{listings}用の設定です.|\lstset|ですべての
% \textsf{listings}の環境の設定をします.
%    \begin{macrocode}
\lstset{%
  language={C},%
  basicstyle={\small\ttfamily},%
  identifierstyle={\small\ttfamily},%
  commentstyle={\small\ttfamily},%
  keywordstyle={\small\ttfamily\bfseries},%
  ndkeywordstyle={\small\ttfamily},%
  stringstyle={\small\sffamily},
  breaklines=true,
  columns=fullflexible,%
  numbers=left,%
  xrightmargin=0zw,%
  xleftmargin=1.5zw,%
  numberstyle={\scriptsize},%
  numbersep=1zw,%
  lineskip=-0.5ex%
}
\ifdraft
\newenvironment{inputex}{\bgroup\small\verbatim}{\endverbatim\egroup}
\newenvironment{inputex*}{\bgroup\small\verbatim}{\endverbatim\egroup}
\else
\lstnewenvironment{inputex}{%
  \lstset{%
     language={[LaTeX]TeX},
     backgroundcolor={\color[cmyk]{0,0,0,\my@default@gray}},%
     frame={tb},
     stepnumber=none,%
  }}{}
\lstnewenvironment{inputex*}{
  \lstset{%
     language={[LaTeX]TeX},
     showspaces,
  }}{}%
\fi
%
\def\lstlistlistingname{ソースコード目次}
\def\lstlistingname{ソースコード}
%    \end{macrocode}
%
% \DescribeMacro{\wrapoverhang}
%
% \textsf{wrapfig} 用にパラメーターの設定をします。
%    \begin{macrocode}
\setlength{\wrapoverhang}{\fullwidth}
\addtolength{\wrapoverhang}{-\textwidth}
\addtolength{\wrapoverhang}{\marginparsep}
\addtolength{\wrapoverhang}{-2zw}
%    \end{macrocode}
%
% \subsection{自作マクロ}
%
% \DescribeMacro{\bs}
% バックスラッシュにアクセスするために|\bs|命令を定義します.
% 様々な場面で使われますがタイプライタ体での出力のときにしか
% 使いません.
%    \begin{macrocode}
%\newcommand*{\bs}{\symbol{'134}}
\let \bs \@backslashchar
\let \rb \@charrb
\let \lb \@charlb
\let \vs \textvisiblespace
%    \end{macrocode}
%
%
% \DescribeMacro{\cmd}
% 文書中で{\LaTeX}のコマンドを参照するときに使われる命令です.
% |\cmd{newcommand}|のように使います.先頭にバックスラッ
% シュが付加されます.これは前述の|\bs|命令が使われます.
% \DescribeMacro{\C}
% {\LaTeX}のコマンドを「索引」に追加する場合は|\cmd|ではなく
% |\C|命令を使うようにします.|\C|はアット`@'を含んでも
% 良いことにしています.
%    \begin{macrocode}
\newcommand*{\cmd}[1]{\texttt{\bs#1}}
% \Cmd is an obsolete command.
%\newcommand*{\Cmd}[1]{%
%   \glossary{#1@\texttt{\hspace*{-1.2ex}{\protect\bs#1}}}%
%   \texttt{\bs#1}}
%
\newcommand*\C[1]{%
  \def\at@mark{@}%
  \let\out@char\@empty
  \@tfor\char@temp:=#1\do{%
     \if\at@mark\char@temp
      \edef\out@char{\out@char"\at@mark}%"
     \else
       \edef\out@char{\out@char\char@temp}%
     \fi}%
  \edef\gloss@hoge{\noexpand\glossary{%
     \out@char\string @\string%
     \hspace*{-1.2ex}\string\texttt{\string\BS\space\out@char}}}%
%  \edef\gloss@hoge{\noexpand\glossary{コマンド!%
%     \out@char\string @\string\hspace*{-1.2ex}%
%     \string\texttt{\string\BS\space\out@char}}}%
  \gloss@hoge %\cindex@hoge
  \texttt{\BS#1}}%
%    \end{macrocode}
%
%
% \DescribeMacro{\env}
% {\LaTeX}の環境型のコマンドを文書中で参照するときは
% |\env|命令を使います.
% \DescribeMacro{\Env}
% その環境を索引に追加したければ|\E|命令を使います.
%    \begin{macrocode}
\newcommand*{\env}[1]{\texttt{#1}}%
\newcommand*{\E}[1]{%
   \index{環境!#1@\texttt{#1}}%
   \glossary{#1かんきよう@\texttt{#1}環境}%
   \texttt{#1}}%
%    \end{macrocode}
%
% \DescribeMacro{\word}
% 語句は|\word|命令の中に入れます.
% \DescribeMacro{\Z}
% 語句を索引に追加するときは|\Z|命令を使います.
%    \begin{macrocode}
\newcommand*{\word}[1]{#1}
\newcommand*{\Z}[1]{\index{#1}#1}
%    \end{macrocode}
%
%
% \DescribeMacro{\zdash}
%  索引中に挿入する倍角ダッシュを|\protect|するために
%  |\zdash|命令を定義しています.
% \DescribeMacro{\indindz}
% \DescribeMacro{\zindind}
%  階層的な索引には「親」と「子」を指定してエントリさせます.
%    \begin{macrocode} 
\DeclareRobustCommand*\zdash{\char\jis"213D\kern-.5zw%"
   \char\jis"213D\kern-.5zw\char\jis"213D\relax}
\newcommand*\zindind[2]{\index{#1!\zdash#2}}
\newcommand*\indindz[2]{\index{#1!#2\zdash}}
%    \end{macrocode}
%
%
% \DescribeMacro{\prog}
% プログラム名を参照する場合は|\prog|命令を使います.
%    \begin{macrocode}
% \DescribeMacro{\Prog}
% プログラム名を索引に追加する場合は|\Prog|命令を使います.
\newcommand*{\prog}{\@ifnextchar[{\yomi@prog}{\@prog}}%]
\newcommand*{\@prog}[1]{#1}
\newcommand*{\yomi@prog}[2][]{#2}
\newcommand*{\Prog}{\@ifnextchar[{\yomi@Prog}{\@Prog}}
\newcommand*{\@Prog}[1]{%
   \index{プログラム!#1}\index{#1}#1}
\newcommand*{\yomi@Prog}[2][]{%
   \index{プログラム!#1@\protect#2}%
   \index{#1@\protect#2}#2}
%    \end{macrocode}
%
%
% \DescribeMacro{\fl}
% ファイル名を参照する場合は|\fl|を使います.
% \DescribeMacro{\Fl}
% ファイル名を参照し索引に追加するときは|\Fl|を使います.
%    \begin{macrocode}
\newcommand*{\fl}[1]{\texttt{#1}}
\newcommand*{\Fl}[1]{%
   \index{ファイル!#1@\texttt{#1}}%
   \index{#1@\texttt{#1}}\texttt{#1}}
%    \end{macrocode}
%
%
%
% \DescribeMacro{\cls}
% しつこいようですが,ドキュメントクラスを参照するときは
% |\cls|命令を使います。
% \DescribeMacro{\D}
% クラスを索引にも追加するときは|\D|命令を使います.
%    \begin{macrocode}
\newcommand*{\cls}[1]{\textsf{#1}}
\newcommand*{\D}[1]{%
   \index{クラス!#1@\textsf{#1}}%
   \index{#1@\textsf{#1}}\textsf{#1}}
%    \end{macrocode}
%
%
% \DescribeMacro{\sty}
% マクロパッケージは|\sty|で参照します.
% \DescribeMacro{\Y}
% さらに索引にも追加するには|\Sty|命令を使います.
%    \begin{macrocode}
\newcommand*{\sty}[1]{\textsf{#1}}
\newcommand*{\Y}[1]{%
   \index{パッケージ!#1@\textsf{#1}}%
   \index{#1@\textsf{#1}}\textsf{#1}}%
%    \end{macrocode}
%
%
% \DescribeMacro{\bst}
% 文献スタイルを参照するときは|\bst|命令を使います.
% \DescribeMacro{\Bst}
% 索引追加は|\Bst|です。
%    \begin{macrocode}
\newcommand*\bibi[1]{%
  \index{#1@\texttt{#1} (\BibTeX)}\texttt{#1}%
  \index{BibTeX@\BibTeX!#1@\texttt{#1}}%
}
\newcommand*\bubu[1]{%
  \index{#1@\texttt{#1}(文献の種類)}\texttt{#1}%
  \index{ぶんけんのしゅるい@文献の種類!#1@\texttt{#1}}%
}
%
\newcommand*{\bst}[1]{\textsf{#1}}
\newcommand*{\Bst}[1]{%
	\index{文献スタイル!#1@\textsf{#1}}%
	\index{#1@\textsf{#1}}\textsf{#1}}%
%    \end{macrocode}
%
% \DescribeMacro{\latexno}
% 索引において\LaTeX が親になるエントリを追加するのに使います。
%    \begin{macrocode}
\newcommand*\latexno[1]{\index{LaTeX@\LaTeX!\zdash #1}}
%    \end{macrocode}
%
% \DescribeMacro{\hito}
% 人名を書くときは|\hito|命令を使い敬称を省略します.
% 敬称は|\hito|命令側で統一します.
% \DescribeMacro{\Hito}
% 索引に追加する場合は|\H|命令です.読みがある場合は
% 任意引数で指定します.しかし、色々と考えた結果、 |\ppl| を基本に使います。
%    \begin{macrocode}
\newcommand*{\hito}{\@ifnextchar[{\yomi@hito}{\@hito}}
\newcommand*{\yomi@hito}[2][]{#2氏}
\newcommand*{\@hito}[1]{#1氏}
\newcommand*{\Hito}{\@ifnextchar[{\yomi@Hito}{\@Hito}}
\newcommand*{\yomi@Hito}[2][]{%
	\index{人名!#1@\protect{#2}}%
	\index{#1@\protect{#2}}#2氏}
\newcommand*{\@Hito}[1]{\index{人名!#1}\index{#1}#1氏}
\newcommand*\ppl[1]{\index{人名!#1}#1氏}
%    \end{macrocode}
%
% \DescribeMacro{\kount}
% {\LaTeX}のカウンタ名を参照する場合は|\kount|を使います.
% \DescribeMacro{\K}
% 案の定索引に追加するには|\K|を使います.
%    \begin{macrocode}
\newcommand*{\kount}[1]{\texttt{#1}}
\newcommand*{\K}[1]{%
  \index{#1@\texttt{#1}\pp{カウンタ}}\index{カウンタ!#1@\texttt{#1}}%
  \texttt{#1}}
%    \end{macrocode}
%
%
% \DescribeMacro{\suf}
% 拡張子を示す場合は|\suf|命令を使います.引数にピリオドは
% 省略します。
% \DescribeMacro{\Suf}
% 索引にも追加する場合は|\Suf|を使います.
%    \begin{macrocode}
\newcommand*{\suf}[1]{\texttt{.#1}}
\newcommand*{\Suf}[1]{%
  \index{拡張子!#1@\texttt{\protect\hspace*{-1ex}.#1}}%
  \index{#1@\texttt{\protect\hspace*{-1ex}.#1} (拡張子)}%
  \texttt{.#1}}
%    \end{macrocode}
%
%
% \DescribeMacro{\va}
% 一般的な「変数」と呼ばれる要素に対しては|\va|命令を使います.
% \DescribeMacro{\Va}
% |\Va|命令は2つの引数を取ります。1つ目には任意の文字列、2つ目には
% 拡張子を書きますから,これは任意のファイルを拡張子付きで
% 示す場合に使います.
% \DescribeMacro{\str}
% ソースコードの入力やその他必要と思われる「文字列」に対しては
% |\str|命令を使います.これは先頭の文字列が|\string|で
% カテゴリーが無効になります.
%    \begin{macrocode}
\newcommand{\va}[1]{{\normalfont$\langle$\mbox{}\textit{#1}\mbox{}$\rangle$}}
\newcommand{\Va}[2]{%
   $\langle$\mbox{}\textit{#1}\mbox{}$\rangle$\suf{#2}}
\newcommand{\str}[1]{{\normalfont\ttfamily\mdseries\string#1}}
%    \end{macrocode}
%
%
% \DescribeMacro{\pa}
% {\LaTeX}コマンドの必須引数を示すために|\pa|命令を使います.
% \DescribeMacro{\opa}
% 任意引数の場合は|\opa|命令を使います.
% \DescribeMacro{\xy}
% 座標系の場合は|\xy|を使い,$x$と$y$の2つの引数を渡します.
%    \begin{macrocode}
\newcommand{\pa}[1]{{\ttfamily\string{}\va{#1}{\ttfamily\string}}}%
\newcommand{\opa}[1]{{\ttfamily[}\va{#1}{\ttfamily]}}%
\newcommand{\xy}[2]{\string({\itshape#1}\texttt, {\itshape#2}\string)}
%    \end{macrocode}
%
%
% \DescribeMacro{\option}
% パッケージオプションやクラスオプションには|\option|命令を使う.
% これはスラント体に変更される.
% \DescribeMacro{\Option}
% オプションを索引に追加する場合には|\Option|命令を使う.
% \DescribeMacro{\copt}
% 特別にプログラムのコマンドラインオプションの場合は|\copt|
% 命令を使うようにすれば良い.これは索引に追加する必要はない.
%    \begin{macrocode}
\newcommand{\option}[1]{\textsl{#1}}
\newcommand{\Option}[1]{%
	\index{#1@\textsl{#1}}%
	\index{オプション!#1@\textsl{#1}}%
	\textsl{#1}}
\newcommand{\copt}[1]{\texttt{#1}}
%    \end{macrocode}
%
%
% \DescribeMacro{\qu}
% \DescribeMacro{\qq}
% 欧文の引用には|\qu|と|\qq|を使います.単語の引用は|\qu|で,
% 文の引用には|\qq|を使うようにします.
% \DescribeMacro{\yo}
% \DescribeMacro{\yy}
% 和文の引用には|\yo|と|\yy|を使います.これも同じように
% 単語には|\yo|で,文には|\yy|です.
% \DescribeMacro{\pp}
% 丸括弧で括る場合は|\pp|命令を使います.全角丸括弧が使われます.
% \DescribeMacro{\wasyo}
% \DescribeMacro{\yousyo}
% 雑誌名や書籍名を参照するときは|\wasyo|と|\yousyo|命令を
% 使います.和書の場合は|\wasyo|,洋書の場合は|\yousyo|です.
% わかりやすいでしょう?
%    \begin{macrocode}
\newcommand*{\qu}[1]{`#1'}
\newcommand*{\qq}[1]{``#1''}
\newcommand*{\yo}[1]{「#1」}
\newcommand*{\yy}[1]{『#1』}
\newcommand*{\pp}[1]{(#1)}
\newcommand*{\wasyo}[1]{『#1』}
\newcommand*{\yousyo}[1]{\emph{#1}}
\newcommand*\Em[1]{{\gtfamily\bfseries#1}}
%    \end{macrocode}
%
%
% \DescribeMacro{\optionlist}
% クラスオプションやパッケージオプションを複数個同時に
% 並べるときは|\optionlist|命令を使います.
% \DescribeMacro{\Optionlist}
% それらを索引に追加する場合は|\Optionlist|になります.
%    \begin{macrocode}
\def\optionlist#1{%
   \@tempcnta=\z@ \@tempcntb=\z@
   \@for\member:=#1\do{\advance\@tempcnta\@ne}%
   \@for\member:=#1\do{\advance\@tempcntb\@ne
      \ifnum\@tempcntb<\@tempcnta
            \textsl{\member},%
      \else
         \ifnum\@tempcntb=\@tempcnta
             \textsl{\member}%
        \fi
      \fi
   }%
}
\def\Optionlist#1{%
   \@tempcnta=\z@ \@tempcntb=\z@
   \@for\member:=#1\do{\advance\@tempcnta\@ne}%
   \@for\member:=#1\do{\advance\@tempcntb\@ne
      \ifnum\@tempcntb<\@tempcnta
         \index{\member @\textsl{\member}}%
         \index{オプション!\member @\textsl{\member}}%
         \textsl{\member},%
      \else
          \ifnum\@tempcntb=\@tempcnta
             \index{\member @\textsl{\member}}%
             \index{オプション!\member @\textsl{\member}}%
             \textsl{\member}%
          \fi%
      \fi
   }%
}
%
%
% \DescribeMacro{\definecolor}
% 基本となるグレーの濃さ.
%    \begin{macrocode}
%\definecolor{mygray}{cmyk}{0,0,0,.8}
\def\my@default@gray{.2}
%    \end{macrocode}
%
%
% \DescribeMacro{\toolbar}
% Windowsのツールバーやメニューバーを参照するときに
% |\toolbar|を使います.途中で改行が起こりませんので|Overfull|
% になるので注意が必要です.
%    \begin{macrocode}
\newcommand*{\toolbar}[1]{{%
   \fboxsep=0pt\fboxrule=0pt\colorbox[cmyk]{0,0,0,\my@default@gray}{[#1]}}} 
%    \end{macrocode}
%
%
% \begin{environment}{myquote}
% jsclasses等では右側が字下げされませんが,
% 右側も字下げする引用環境です.
%    \begin{macrocode}
\newenvironment{myquote}%
  {\list{}{\rightmargin\leftmargin}\item\relax}{\endlist}
%    \end{macrocode}
% \end{environment}
% \begin{environment}{myquotation}
% こちらは段落の行頭が字下げされます.
%    \begin{macrocode}
\newenvironment{myquotation}{%
  \list{}{%
    \listparindent\parindent
    \itemindent\listparindent
    \rightmargin\leftmargin}%
  \item\relax}{\endlist}
%    \end{macrocode}
% \end{environment}
%
%
% \DescribeMacro{\m}
% 以下|\m|,|\M|,|\T|,|\KM|,|\A|,|\B|,|\W|は
% すべて表中で使います.これは左側の要素に記号を出力し
% 右側の要素にソースを出力します.要するに記号の
% 入出力の対を示すために何度も使われています.
% 上記の命令ではバックスラッシュは省略します.
% 結果的に記号が出力されていない場合もあるので
% タイプミスに注意してください.
% \DescribeMacro{\M}
% |\m|の場合は索引に追加しない数学記号であり,
% |\M|命令は数学記号として命令索引に追加します.
% \DescribeMacro{\T}
% 文字記号は|\T|として命令索引に追加します.
% \DescribeMacro{\KM}
% 「記号 命令」は何度も使われるかもしれないので|\KM|
% としてあります.
%    \begin{macrocode}
\newcommand*{\m}[1]{$#1$&\texttt{\string#1}}
\newcommand*{\M}[1]{%
   \glossary{#1@\hspace*{-1.2ex}\texttt{\protect\bs#1}%
	\hskip1em($\protect\csname #1\endcsname$)}%
   $\csname #1\endcsname$&\texttt{\bs#1}}
\newcommand*{\BM}[1]{%
   \glossary{#1@\hspace*{-1.2ex}\texttt{\protect\bs#1}%
	\hskip1em($\protect\csname #1\endcsname$)}%
   \texttt{\bs#1}}
\newcommand*{\T}[1]{%
   \glossary{#1@\hspace*{-1.2ex}\texttt{\protect\bs\string#1}%
	\hskip1em(\csname#1\endcsname)}%
   \csname#1\endcsname&\texttt{\bs\string#1}}
\newcommand*{\KM}{記号&命令}
%    \end{macrocode}
%
% \DescribeMacro{\A}
% 文章中の特殊記号には|\A|を使います.
% \DescribeMacro{\B}
% 文長中のアクセントには|\B|命令を使い,1つ目の引数に
% 命令を,2つ目の引数にアクセントつけられる文字を書きます.
% \DescribeMacro{\W}
% 数式中のアクセントには|\W|命令で,1つ目の引数に
% 命令を,2つ目の引数にアクセントをつけられる記号を書きます.
%    \begin{macrocode}
\newcommand*{\A}[1]{%
   \glossary{#1@\hspace*{-1.2ex}\texttt{\protect\bs#1}%
      \hskip1em(\csname#1\endcsname)}%
   \csname#1\endcsname&\texttt{\bs#1}}
\newcommand*{\B}[2]{%
   \glossary{#1@\hspace*{-1.2ex}\texttt{\protect\bs#1}%
      \hskip1em(\csname#1\endcsname{#2}\relax)}%
   \csname#1\endcsname{#2}&%
    \texttt{\protect\bs\string#1\string{#2\string}}}
% \W{#1}{#2} 数式中のアクセント
\newcommand*{\W}[2]{%
  \glossary{#1@\hspace*{-1.2ex}\texttt{\bs#1}%
    \hskip1em($\csname#1\endcsname{#2}$)}%
  $\csname#1\endcsname{#2}$ & %
  \texttt{\bs#1\string{#2\string}}}%
%    \end{macrocode}
%
% \DescribeMacro{\demowidth}
% 文章中で線の長さを示すには|\demowidth|を使います.これは
% lshortからの改変です.長さが負の場合の処理を追加した方が
% 良いかもしれません.
%    \begin{macrocode}
\newcommand*{\demowidth}[1]{%
   \rule{0.3pt}{1.3ex}\rule{#1}{0.3pt}\rule{0.3pt}{1.3ex}}
%    \end{macrocode}
%
%
% どこかで|hoge|カウンタが使われています.というか様々な
% ところでこの|hoge|カウンタが乱用されています.
%    \begin{macrocode}
\newcounter{hoge}
\newcommand*{\bool}{ブール値}%?
%    \end{macrocode}
%
% \begin{environment}{Prob}
% 「問題」型の環境です.\textsf{theorem}パッケージを使って
% 定義しています.
%    \begin{macrocode}
\theoremheaderfont    {\reset@font \headfont}
\theorembodyfont      {\reset@font \rmfamily}
\theoremstyle         {plain}
\newtheorem{Prob}     {\mantriangleright 問題}[chapter]
\newtheorem{Exe}[Prob]{$\triangleright$ 例題}
\newenvironment{Trick}%
  {\begin{list}{\mbox{\normalsize\dbend}}{%
  \leftmargin = 3zw
  \labelwidth = 2zw
  \rightmargin = 0zw
  \small}\item \relax}{\end{list}}
%    \end{macrocode}
% \end{environment}
%
%
% \subsection{リスト表示・ベタ書きなどなど}
%
% \begin{environment}{InText}
% ユーザが入力すべきテキストを示すには|InText|環境を
% 使います.これは|verbatim|環境に|\small|と左側の字下げをした
% だけで,至ってシンプルです.
%    \begin{macrocode}
\newenvironment{InText}{%
  \list{}{\leftmargin=2zw \rightmargin=0zw}
   \item\small\verbatim}{\endverbatim \endlist}%
%    \end{macrocode}
% \end{environment}
%
% \begin{environment}{OutText}
% ソースコードをタイプセットしたあとの表示を示すには
% |OutText|環境を使います.これは入力の|InText|と対になります.
%    \begin{macrocode}
\newenvironment{OutText}{\unitlength=1pt
	\begin{screen}}{\end{screen}}
%    \end{macrocode}
% \end{environment}
%
% \DescribeMacro{\type}
% \DescribeMacro{\key}
% one-line のコンソールからの入力には |\type| 命令を使います。
% キートップを出力するには |\key{key1, key2,..., keyn}| を使います。
% \begin{environment}{InTerm}
% コンソールからの文字列の入力には|InTerm|環境を
% 使います.複数行の入力には|\item|命令を先頭におきます.
%    \begin{macrocode}
%   {\list{\rule[-.2em]{1ex}{1em}}{\ttfamily
%      \itemsep=-.5ex \parsep=-.5ex}\item\relax}%
%   {\endlist}
\newif\if@TYPE
\newcommand*\type{\@ifnextchar[{\@@type}{\@type}}
\def\@hoge{\begingroup \urlstyle{tt}\Url}
\def\@@type[#1]{\if@TYPE\item[{\ttfamily#1}]\fi
   \begingroup \urlstyle{tt}\Url}
\def\@type#1{\if@TYPE \item\@hoge{#1}%
      \else\underline{\@hoge{#1}}\fi}
%
\newcommand*\key{\thinspace\@key}%
\def\@key#1{\@tempcnta=\z@%
 \@for\member:=#1\do{%
   \ifnum\@tempcnta<1%
      \keytop{\member}%
   \else%
      \texttt{+}\keytop{\member}%
   \fi%
   \advance\@tempcnta\@ne}%
 \thinspace}%
%
\newenvironment{InTerm}{\@TYPEtrue
   \list{\mbox{\texttt\$}}{%
      \rightmargin=0pt
      \itemsep=-.5ex 
      \parsep=-.5ex}}%
   {\endlist \@TYPEfalse}
%
%    \end{macrocode}
% \end{environment}
%
% \begin{environment}{OutTerm}
% コンソールに出力される文字列を示す場合は|OutTerm|環境を
% 使います。draftオプションが有効のときは|verbatim|に入れます.
% そうでない時はlistingsパッケージを使います.
%    \begin{macrocode}
\ifdraft
\newenvironment{OutTerm}{%
   \list{}{\leftmargin=1.5zw \rightmargin=1.5zw}
   \item\small\verbatim}{\endverbatim \endlist}%
\else
\lstnewenvironment{OutTerm}{%
  \lstset{%
    columns=[l]{fullflexible},
    basicstyle={\small\ttfamily},%
    identifierstyle={\small},%
    commentstyle={\small},%
    keywordstyle={\small\bfseries},%
    ndkeywordstyle={\small},%
    numbers=none,%
    formfeed=\linebreak,%
    frameshape={yny}{}{}{yny},
    xrightmargin=1.5zw,%
    xleftmargin=1.5zw,%
    }}{}%
\fi%
%    \end{macrocode}
% \end{environment}
%
%
% \begin{environment}{Syntax}
% {\LaTeX}などにおける重要な構文は|Syntax|環境中に入れます.
% 改行などは行われず,ただ単に要素を文章幅いっぱいの枠付きの
% 箱に挿入するだけですから,ページ区切りには気をつけてください.
%    \begin{macrocode}
\newenvironment{Syntax}%
  {\bgroup \parindent=0pt%
    \par\addvspace{1ex plus 0.8ex minus 0.5ex}%
    \vskip-\parskip\begin{Sbox}%
    \begin{minipage}{\ftextwidth}}%
  {\end{minipage}\end{Sbox}\fbox{\TheSbox}%\IOlabel%
    \par\addvspace{1ex plus 0.8ex minus 0.5ex}%
    \vskip-\parskip\egroup\par\noindent\ignorespacesafterend}%
%    \end{macrocode}
% \end{environment}
%
%
% \subsection{相互参照}
%
% \DescribeMacro{\pref}
% ページを参照する場合は|\pageref|ではなく|\pref|を使います.
% この場合|\pref|側で「○ページ」の「ページ」に該当する
% 文字を統一します.
% \DescribeMacro{\chapref}
% 章を参照するときは|\ref|ではなく|\chapref|命令を使います.
% \DescribeMacro{\secref}
% 同じように節も|\secref|を使います.%
% \DescribeMacro{\fullref}
% ページ番号と節番号の両方を「○ページ○節」として
% 参照するには|\fullref|を使います.
% \DescribeMacro{\figref}
% 図は|\figref|です.
% \DescribeMacro{\tabref}
% 表は|\tabref|です.
% \DescribeMacro{\eqref}
% 式を参照するための|\eqref|はすでにamsmathか何かで定義されているので
% |\def|を使っています.
%    \begin{macrocode}
\newcommand*{\pref}[1]{\pageref{#1}~ページ}
\newcommand*{\chapref}[1]{\ref{#1}~章}
\newcommand*{\secref}[1]{\ref{#1}~節}
\newcommand*{\fullref}[1]{\pref{#1}\secref{#1}~参照}
\newcommand*{\figref}[1]{\figurename~\ref{#1}}
\newcommand*{\tabref}[1]{\tablename~\ref{#1}}
\def\eqref#1{式~(\ref{#1})}
%    \end{macrocode}
%
%
% \subsection{ロゴ}
%
% \DescribeMacro{\Xy}
% 各種ロゴを定義します.
% \DescribeMacro{\PIC}
% \DescribeMacro{\Tpic}
% \DescribeMacro{\JLaTeXe}
% \DescribeMacro{\XyMTeX}
% \DescribeMacro{\eTeX}
% \DescribeMacro{\eLaTeX}
% \DescribeMacro{\AmsLaTeX}
% \DescribeMacro{\dvipdfmx}
% dvipdfm$x$のロゴは正式なものではないかもしれません.
%    \begin{macrocode}
\DeclareRobustCommand*{\Xy}{%
   \leavevmode\hbox{\kern-.1em X\kern-.3em\lower.4ex\hbox{Y\kern-.15em}}%
   \xspace
}
\DeclareRobustCommand*{\PIC}{PIC\xspace}
\DeclareRobustCommand*{\Tpic}{{\normalfont\textsc{Tpic}}\xspace}
\DeclareRobustCommand*{\JLaTeXe}{%
   \leavevmode%
   \lower.5ex\hbox{\rm J}\kern-.1em\LaTeXe\xspace
}
\DeclareRobustCommand*{\XyMTeX}{%
   X\kern-.3em\smash{\raise.5ex\hbox{$\m@th\Upsilon$}}%
   \kern-.3em{M}\kern-0.1em\TeX\xspace
}
\DeclareRobustCommand*{\eTeX}{$\m@th\varepsilon$-\TeX\xspace}
\DeclareRobustCommand*{\eLaTeX}{$\m@th\varepsilon$-\LaTeX\xspace}
\DeclareRobustCommand*\NTS{%
   \leavevmode\hbox{$\cal N\kern-0.35em\lower0.5ex\hbox{$\cal T$}%
   \kern-0.2emS$}\xspace
}
\DeclareRobustCommand*{\AmSLaTeX}{\AmS-\LaTeX\xspace}
\DeclareRobustCommand*{\dvipdfmx}{%
   Dvipdfm{\rmfamily\itshape x}\xspace
}
\DeclareRobustCommand*\XeTeX{\leavevmode
  \setbox0=\hbox{X\lower.5ex\hbox{\kern-.15em\reflectbox{E}}%
  \kern-.1667em \TeX}\dp0=0pt\ht0=0pt\box0}
\DeclareRobustCommand*\XeLaTeX{\leavevmode
  \setbox0=\hbox{X\lower.5ex\hbox{\kern-.15em\reflectbox{E}}%
  \kern-.0833em \LaTeX}\dp0=0pt\ht0=0pt\box0}
%    \end{macrocode}
%
% \subsection{語句}
%
% \DescribeMacro{\LMANUAL}
% \DescribeMacro{\COMP}
% \DescribeMacro{\GCOMP}
% \DescribeMacro{\WCOMP}
% \DescribeMacro{\TEXBOOK}
% \DescribeMacro{\METAFONTBOOK}
%
% 用語の統一のためにいくつかの書籍名や用語を登録しています.
% |\LMANUAL|,|\COMP|,|\GCOMP|,|\WCOMP|の4冊は金字塔の作品.
% |\TEXBOOK|と|\METAFONTBOOK|はKnuth氏のバイブル.
%    \begin{macrocode}
\newcommand*{\LMANUAL}{文書処理システム{\LaTeXe}}
\newcommand*{\COMP}{{\LaTeX}コンパニオン}
\newcommand*{\GCOMP}{{\LaTeX}グラフィックスコンパニオン}
\newcommand*{\WCOMP}{{\LaTeX} Webコンパニオン}
\newcommand*{\TEXBOOK}{\TeX book\xspace}% 原著
\newcommand*{\METAFONTBOOK}{\MF ブック} % 訳著
\newcommand*{\ANOTHERMANUAL}{Another Manual\xspace}
\newcommand*{\PS}{PostScript\xspace}
\newcommand*{\IM}{ImageMagick\xspace}
\newcommand*{\GS}{Ghostscript\xspace}
\newcommand*{\BB}{BoundingBox\xspace}
\newcommand*{\laTEX}{\TeX/\LaTeX\xspace}
%    \end{macrocode}
%
%
% \DescribeMacro{\dos}
% 端末に表示される警告などを示すために|\dos|命令を使います.
% これは背景が黒,文字色が白になりますので,まるでプロンプトの
% ようなスタイルになります.
% \DescribeMacro{\dosh}
% |\dosh|を使うと文章幅いっぱいのスタイルになります。自分で
% 適宜改行をします.
%    \begin{macrocode}
\newcommand{\dos}[1]{%
   \colorbox{black}{\color{white}{%
   \small\normalfont\ttfamily #1\hfil}}}
\newcommand{\dosh}[1]{%
   \noindent\colorbox{black}{%
   \hbox to \ftextwidth{\color{white}{%
   \small\normalfont\ttfamily #1\hfil}}}}
%    \end{macrocode}
%
%
% \DescribeMacro{\IOmargin}
% \DescribeMacro{\IOlabel}
% 文章幅からはみ出る要素は一応版面いっぱいまでなら許容される.
% このとき,奇数ページか偶数ページかでマージンを変更する.
% これには|\IOmargin|と|\IOlabel|を合わせて使うようにすると
% 可能です.文章幅を飛び出る要素の直前に|\IOmargin|命令を書き
% 要素の直後に|\IOlabel|を書きます.あらかじめ要素は|\makebox|
% 命令などで幅を0ptに見せかける処理が必要になります.\par
% 入出力の対を版面いっぱいに表示するために|InOut|環境を定義します.
% これもlshortの定義を少し変更しただけです.\textsf{fancybox}の
% マクロを使えばもう少し簡単になる?
%    \begin{macrocode}
\newlength{\IOm}
\setlength{\IOm}{\textwidth}
\addtolength{\IOm}{-\fullwidth}
%
\newwrite\example@out
%
\newcounter{IOcnt}
\setcounter{IOcnt}{1}
%
\newcommand{\IOmargin}{%
  \stepcounter{IOcnt}%
  \expandafter\ifx\csname r@exa:\theIOcnt\endcsname\relax
  \else
    \ifHyper
      \ifodd\HyPsd@pageref{exa:\theIOcnt}%
         \hspace*{0pt}%
      \else
         \hspace*{\IOm}%
      \fi
    \else
      \ifodd\pageref{exa:\theIOcnt}%
         \hspace*{0pt}%
      \else
         \hspace*{\IOm}%
      \fi
    \fi
  \fi
}
\newcommand{\IOlabel}{\label{exa:\theIOcnt}}
\newenvironment{InOut}%
   {\begingroup%
      \@bsphack%
       \immediate\openout \example@out \jobname.exa%
       \let\do\@makeother\dospecials\catcode`\^^M\active%
       \def\verbatim@processline{%
          \immediate\write\example@out{\the\verbatim@line}}%
          \verbatim@start}%
   {\immediate\closeout\example@out\@esphack\endgroup%
   \stepcounter{IOcnt}% 
   \setlength{\parindent}{0pt}%
   \par\addvspace{3.0ex plus 0.8ex minus 0.5ex}\vskip-\parskip%
\expandafter\ifx\csname r@exa:\theIOcnt\endcsname \relax
\else
   \ifHyper
      \ifodd\HyPsd@pageref{exa:\theIOcnt}%
         \hspace*{0pt}%
      \else
         \hspace*{\IOm}%
      \fi
   \else
      \ifodd\pageref{exa:\theIOcnt}%
         \hspace*{0pt}%
      \else
         \hspace*{\IOm}%
      \fi
   \fi
\fi
   \makebox[0pt][l]{%
   {\begin{minipage}[c]{.47\fullwidth}%
      \small\verbatiminput{\jobname.exa}%
   \end{minipage}}%
   \hspace{0.05\fullwidth}%
   {\begin{minipage}{.47\fullwidth}%
      \begin{trivlist}\item\small\input{\jobname.exa}%
      \end{trivlist}%
   \end{minipage}}%
   }\label{exa:\theIOcnt}%
   \par\addvspace{3.0ex plus 0.8ex minus 0.5ex}\vskip-\parskip}%
%    \end{macrocode}
%
%
% \DescribeMacro{\image}
% その他色々
%    \begin{macrocode}
\newcommand{\image}[4][]{%
\begin{figure}[htbp]
 \begin{center}
   \includegraphics[#1]{images/#2}
   \caption{#3}\label{fig:#4}
 \end{center}
\end{figure}}
%\newenvironment{ftable}[1][htbp]%
%  {\begin{table}[#1]
%   \begin{Sbox}\begin{minipage}{%
%  (\linewidth-2\fboxrule-2\fboxsep)}}%
%  {\end{minipage}\end{Sbox}\fbox{\TheSbox}\end{table}} 
%
\newcommand*{\kutiref}[1]{口絵~\ref{kuti:#1}}
\newcommand*{\kuti}[1]{\refstepcounter{enumi}口絵\theenumi~#1}
\newenvironment{column}{\begin{small}\paragraph*{コラム}}{\end{small}}
\DeclareRobustCommand*{\pdfTeX}{pdf\TeX\xspace}
\DeclareRobustCommand*{\pdfeTeX}{pdf\eTeX\xspace}
\DeclareRobustCommand*{\pdfLaTeX}{pdf\LaTeX\xspace}
\DeclareRobustCommand*{\pdfeLaTeX}{pdf\eLaTeX\xspace}
\DeclareRobustCommand*{\Context}{Con{\TeX}t\xspace}
\DeclareRobustCommand*{\texinfo}{texinfo\xspace}
\DeclareRobustCommand*{\txfonts}{\texttt{TX}\textsf{Fonts}\xspace}
\DeclareRobustCommand*{\pxfonts}{\texttt{PX}\textsf{Fonts}\xspace}
\newcommand*{\myKuten}{。}
\newcommand*{\myTouten}{、}
%    \end{macrocode}
%
%
% \DescribeMacro{\DeclareFontShape}
% 不足するシリーズを宣言します。
%    \begin{macrocode}
\DeclareFontShape{JY1}{gt}{m} {sc}{<->ssub*gt/m/n}{}
\DeclareFontShape{JT1}{gt}{m} {sc}{<->ssub*gt/m/n}{}
\DeclareFontShape{JY1}{gt}{bx}{it}{<->ssub*gt/m/n}{}
\DeclareFontShape{JT1}{gt}{bx}{it}{<->ssub*gt/m/n}{}
\DeclareFontShape{JT1}{gt}{bx}{sl}{<->ssub*gt/m/n}{}
\DeclareFontShape{JY1}{gt}{bx}{sl}{<->ssub*gt/m/n}{}
\DeclareFontShape{JY1}{mc}{m} {ui}{<->ssub*mc/m/n}{}
\DeclareFontShape{JT1}{mc}{m} {ui}{<->ssub*mc/m/n}{}
%    \end{macrocode}
%

% \subsection{目次など}
%
% \DescribeMacro{\tableofcontents}
% \DescribeMacro{\l@chapter}
% \DescribeMacro{\l@figure}
% \DescribeMacro{\listoffigures}
% \DescribeMacro{\listoftables}
% \DescribeMacro{\l@lstlisting}
% |tocdepth| は 2 にします。
% jsbook のバージョンによっては |\l@figure| が 欧文用のままです。
%    \begin{macrocode}
\setcounter{tocdepth}{2}
%
\renewcommand{\tableofcontents}{%
  \if@twocolumn
    \@restonecoltrue\onecolumn
  \else
    \@restonecolfalse
  \fi
  \chapter*{\contentsname%
 	\@mkboth{\contentsname}{\contentsname}%
  	\@ifundefined{texorpdfstring}{}{%
           \pdfbookmark{\contentsname}{contents}}%
  	}%
  \@starttoc{toc}%
  \if@restonecol\twocolumn\fi
}
%
\renewcommand{\l@chapter}[2]{%
  \ifnum \c@tocdepth >\m@ne
    \addpenalty{-\@highpenalty}%
    \addvspace{1.0em \@plus\p@}
    \begingroup
      \parindent\z@
      \rightskip\@tocrmarg
      \parfillskip-\rightskip
      \leavevmode\large\headfont
      \setlength\@lnumwidth{4.683zw}%
      \advance\leftskip\@lnumwidth \hskip-\leftskip
      \hb@xt@\z@{\color[cmyk]{0,0,0,\my@default@gray}\vrule height 1em width 4pt depth1em\hss}%
      \hskip 8pt #1\nobreak\hfill\nobreak\hb@xt@\@pnumwidth{\hss#2}\par
        {\color[cmyk]{0,0,0,\my@default@gray}\hrule depth 4pt}\par\nobreak\vskip8pt
      \penalty\@highpenalty
    \endgroup
  \fi}
%
\renewcommand*{\l@figure}{\@dottedtocline{1}{1zw}{3.683zw}}
%
\renewcommand{\listoffigures}{%
  \if@twocolumn\@restonecoltrue\onecolumn
  \else\@restonecolfalse\fi
  \chapter*{\listfigurename
  \@mkboth{\listfigurename}{\listfigurename}}%
  \@starttoc{lof}%
  \if@restonecol\twocolumn\fi
}
%
\renewcommand{\listoftables}{%
  \if@twocolumn\@restonecoltrue\onecolumn
  \else\@restonecolfalse\fi
  \chapter*{\listtablename
  \@mkboth{\listtablename}{\listtablename}}%
  \@starttoc{lot}%
  \if@restonecol\twocolumn\fi
}
%
\let\l@lstlisting\l@figure
%    \end{macrocode}
% 
% \DescribeMacro{\setcounter}
% 目次のレベルと番号づけの深さを設定します。
%    \begin{macrocode}
\setcounter{secnumdepth}{2}
\setcounter{tocdepth}{1}
%    \end{macrocode}
%
%
% \subsection{索引・参考文献}
%
% \DescribeMacro{\makeindex}
% \DescribeMacro{\makeglossary}
% \DescribeMacro{\indexname}
% \DescribeMacro{\glossaryname}
% 索引と命令索引を作るために|\makeindex|と|\makeglossary|
% しています。それに因んで|\indexname|と|\glossaryname|を
% それぞれ「索 引」と「命 令 索 引」に決めています.
%    \begin{macrocode}
\makeindex
\makeglossary
\def\indexname{索 引}
\def\glossaryname{命 令 索 引}
%    \end{macrocode}
%
% \begin{environment}{theglossary}
% \begin{environment}{theindex}
% \begin{environment}{thebibliography}
% 
% 命令索引を出力するには既存の |glossary| 系統のものを使います。
%    \begin{macrocode}
\renewenvironment{theindex}{% 索引を3段組で出力する環境
    \if@twocolumn
      \onecolumn\@restonecolfalse
    \else
      \clearpage\@restonecoltrue
    \fi
    \columnseprule.4pt \columnsep 2zw
    \ifx\multicols\@undefined
      \twocolumn[\@makeschapterhead{\indexname}%
        \addcontentsline{toc}{chapter}{\indexname}]%
    \else
      \ifdim\textwidth<\fullwidth
        \setlength{\evensidemargin}{\oddsidemargin}
        \setlength{\textwidth}{\fullwidth}
        \setlength{\linewidth}{\fullwidth}
        \begin{multicols}{3}[\chapter*{\indexname}%
        \addcontentsline{toc}{chapter}{\indexname}]%
      \else
        \begin{multicols}{2}[\chapter*{\indexname}%
        \addcontentsline{toc}{chapter}{\indexname}]%
      \fi
    \fi
    \@mkboth{\indexname}{\indexname}%
    \plainifnotempty % \thispagestyle{plain}
    \parindent\z@
    \parskip\z@ \@plus .3\p@\relax
    \let\item\@idxitem
    \raggedright
    \small\narrowbaselines\normalfont
  }{
    \ifx\multicols\@undefined
      \if@restonecol\onecolumn\fi
    \else
      \end{multicols}
    \fi
    \clearpage
  }
%
\newenvironment{theglossary}{% 命令索引を3段組で出力する環境
    \if@twocolumn
      \onecolumn\@restonecolfalse
    \else
      \clearpage\@restonecoltrue
    \fi
    \columnseprule.4pt \columnsep 2zw
    \ifx\multicols\@undefined
      \twocolumn[\@makeschapterhead{\glossaryname}%
      \addcontentsline{toc}{chapter}{\glossaryname}%
      ]%
    \else
      \ifdim\textwidth<\fullwidth
        \setlength{\evensidemargin}{\oddsidemargin}
        \setlength{\textwidth}{\fullwidth}
        \setlength{\linewidth}{\fullwidth}
        \begin{multicols}{3}[\chapter*{\glossaryname}%
        \addcontentsline{toc}{chapter}{\glossaryname}%
        ]%
      \else
        \begin{multicols}{2}[\chapter*{\glossaryname}%
        \addcontentsline{toc}{chapter}{\glossaryname}%
        ]%
      \fi
    \fi
    \@mkboth{\glossaryname}{\glossaryname}%
    \plainifnotempty % \thispagestyle{plain}
    \parindent\z@
    \parskip\z@ \@plus .3\p@\relax
    \let\item\@idxitem
    \raggedright
    \footnotesize\narrowbaselines
  }{
    \ifx\multicols\@undefined
      \if@restonecol\onecolumn\fi
    \else
      \end{multicols}
    \fi
    \clearpage
  }%
\renewenvironment{thebibliography}[1]{%
  \global\let\presectionname\relax
  \global\let\postsectionname\relax
  \chapter*{\bibname\@mkboth{\bibname}{\bibname}}%
      \addcontentsline{toc}{chapter}{\bibname}%
   \list{\@biblabel{\@arabic\c@enumiv}}%
        {\settowidth\labelwidth{\@biblabel{#1}}%
         \leftmargin\labelwidth
         \advance\leftmargin\labelsep
         \@openbib@code
         \usecounter{enumiv}%
         \let\p@enumiv\@empty
         \renewcommand\theenumiv{\@arabic\c@enumiv}}%
   \sloppy
   \clubpenalty4000
   \@clubpenalty\clubpenalty
   \widowpenalty4000%
   \sfcode`\.\@m}
  {\def\@noitemerr
    {\@latex@warning{Empty `thebibliography' environment}}%
   \endlist
}
%    \end{macrocode}
% \end{environment}
% \end{environment}
% \end{environment}
%
%
% \DescribeMacro{\iiiemdash}
% \DescribeMacro{\bibitem}
% \DescribeMacro{\temp@str}
% 参考文献一覧では著者名が重複した場合 3\,em dash にします。
% 
%    \begin{macrocode}
\DeclareRobustCommand*\iiiemdash{%
  ---\kern-.5em---\kern-.5em---\kern-.5em---\kern-.5em---}
%
\let\orig@bibitem\bibitem
\let\temp@str\@empty
\global\def\bibitem#1#2\newblock{%
   \orig@bibitem{#1}% 
   \def\temp@str{#2} 
   \ifx\temp@str\previous@str 
      \iiiemdash.\space\newblock 
   \else 
      #2\newblock 
   \fi 
   \def\previous@str{#2}% 
}
%    \end{macrocode}

% \subsection{見出し}
%
% \DescribeMacro{\@chapter}
% \DescribeMacro{\@makechapterhead}
% \DescribeMacro{\@makeschapterhead}
%
% 見出し関連は好みで変更しています。
%    \begin{macrocode}
\def\@chapter[#1]#2{%
  \ifnum \c@secnumdepth >\m@ne
    \if@mainmatter
      \refstepcounter{chapter}%
%      \typeout{\@chapapp\thechapter\@chappos}%
      \addcontentsline{toc}{chapter}%
        {\protect\numberline{\@chapapp\thechapter\@chappos}#1}%
    \else\addcontentsline{toc}{chapter}{#1}\fi
  \else
    \addcontentsline{toc}{chapter}{#1}%
  \fi
  \chaptermark{#1}%
  \addtocontents{lof}{\protect\addvspace{10\p@}}%
  \addtocontents{lot}{\protect\addvspace{10\p@}}% 
  \addtocontents{lol}{\protect\addvspace{10\p@}}% listings 用
  \if@twocolumn
    \@topnewpage[\@makechapterhead{#2}]%
  \else
    \@makechapterhead{#2}%
    \@afterheading
  \fi}
% chapter#1 style
\def\@makechapterhead#1{%
  \vspace*{2\Cvs}% 
  {\parindent \z@ \raggedright \normalfont
    \ifnum \c@secnumdepth >\m@ne
      \if@mainmatter
	\hb@xt@\fullwidth{\hfill%
          \huge\headfont\@chapapp{\Huge\bfseries\thechapter}\@chappos}%
        \par\nobreak
	\vskip\Cvs %
      \fi
    \fi
    \interlinepenalty\@M
    \hb@xt@\fullwidth{\hfill\Huge\headfont#1}
    \par\nobreak\vskip\fboxsep
    \hrule height 1ex
    \par\vskip2\Cvs}}%
% \chapter*#1 style 星付き章見出しの変更
\def\@makeschapterhead#1{%
  \vspace*{2\Cvs}% 欧文は50pt
  {\parindent \z@ \raggedright
    \normalfont
    \interlinepenalty\@M
    \hb@xt@\fullwidth{\hfill\Huge \headfont #1}\par\nobreak\vskip\fboxsep
    \hrule height 1ex
    \par\vskip2\Cvs}}% 欧文は40pt
%    \end{macrocode}

% \DescribeMacro{\@sect}
% \DescribeMacro{\@xsect}
% \DescribeMacro{\@ssect}
% 節の見出しに関してもちょっと体裁を変更します。
%    \begin{macrocode}
% \section style
\def\@sect#1#2#3#4#5#6[#7]#8{%
  \ifnum #2>\c@secnumdepth
    \let\@svsec\@empty
  \else
    \refstepcounter{#1}%
    \protected@edef\@svsec{\@seccntformat{#1}\relax}%
  \fi
  \@tempskipa #5\relax
  \ifdim \@tempskipa<\z@
    \def\@svsechd{%
      #6{\hskip #3\relax
      \@svsec #8}%
      \csname #1mark\endcsname{#7}%
      \addcontentsline{toc}{#1}{%
        \ifnum #2>\c@secnumdepth \else
          \protect\numberline{\csname the#1\endcsname}%
        \fi
        #7}}% 目次にフルネームを載せるなら #8
  \else
    \begingroup
      \interlinepenalty \@M % 下から移動
%%%%%%
    \ifnum #2=\@ne
      \colorbox[cmyk]{0,0,0,\my@default@gray}{%
        \hb@xt@\ftextwidth{#6{\@hangfrom{\@svsec}#8}%
          \hfil}%
	}\@@par%
    \else
        \ifnum #2=\tw@
           {\large$\blacktriangledown$\hskip3pt}%
           #6{\@hangfrom{\hskip #3\relax\@svsec}%
              \relax#8\@@par}%
        \else
           #6{\@hangfrom{\hskip #3\relax\@svsec}#8\@@par}%
        \fi
    \fi
    \endgroup
%%%%%
    \csname #1mark\endcsname{#7}%
    \addcontentsline{toc}{#1}{%
      \ifnum #2>\c@secnumdepth \else
        \protect\numberline{\csname the#1\endcsname}%
      \fi
      #7}% 目次にフルネームを載せるならここは #8
  \fi
  \@xsect{#5}}
\def\@xsect#1{%
  \@tempskipa #1\relax
  \ifdim \@tempskipa<\z@
    \@nobreakfalse
    \global\@noskipsectrue
    \everypar{%
      \if@noskipsec
        \global\@noskipsecfalse
       {\setbox\z@\lastbox}%
        \clubpenalty\@M
        \begingroup \@svsechd \endgroup
        \unskip
        \@tempskipa #1\relax
        \hskip -\@tempskipa
      \else
        \clubpenalty \@clubpenalty
        \everypar{\everyparhook}%
      \fi\everyparhook}%
  \else
    \par \nobreak
    \vskip \@tempskipa
    \@afterheading
  \fi
  \par  % 2000-12-18
  \ignorespaces}
\def\@ssect#1#2#3#4#5{%
  \@tempskipa #3\relax
  \ifdim \@tempskipa<\z@
    \def\@svsechd{#4{\hskip #1\relax #5}}%
  \else
    \begingroup
      #4{%
        \@hangfrom{\hskip #1}%
          \interlinepenalty \@M #5\@@par}%
    \endgroup
  \fi
  \@xsect{#3}}
%    \end{macrocode}
%
%
% \DescribeMacro{\paragraph}
% 段落レベルの |\paragraph| はちょっと再定義します。
%    \begin{macrocode}
\if@twocolumn
  \renewcommand{\paragraph}{\@startsection{paragraph}{4}{\z@}%
    {\z@}{-1zw}% 改行せず 1zw のアキ
    {\normalfont\normalsize\headfont}}
\else
  \renewcommand{\paragraph}{\@startsection{paragraph}{4}{\z@}%
    {0.5\Cvs \@plus.5\Cdp \@minus.2\Cdp}%
    {-1zw}% 改行せず 1zw のアキ
    {\normalfont\normalsize\headfont}}
\fi
%    \end{macrocode}
%
% \begin{environment}{abstract}
% abstract 環境の変更をします。これは章の直後に使います。
%    \begin{macrocode}
\renewenvironment{abstract}{%
  \thispagestyle{plainhead}%
  \begin{list}{}{%
    \linewidth=\fullwidth
    \gtfamily\sffamily
    \listparindent=1zw
    \itemindent=\listparindent
    \leftmargin=.3\textwidth
    \rightmargin=0pt
    }\item[]}{\end{list}}%
%    \end{macrocode}
% \end{environment}
%
%
% \subsection{爪かけなど}
% 
% \DescribeMacro{\rtume}
% 爪かけをだすための命令は |\rtume| となります。
% \DescribeMacro{\ps@myhead}
% \DescribeMacro{\ps@plainhead}
% \DescribeMacro{\ps@plain}
% 実際に |\rtume| が使われるのは |\ps@myhead| などになります。
% |\ps@plainhead| は章見出しのあるページになります。
% |\ps@plain| は \ldots 忘れました。
%
%    \begin{macrocode}
\let\draft@mark\@empty
\newif\iftume
%
\newcommand{\rtume}{
\iftume
 \setlength{\unitlength}{10mm}%
 \begin{picture}(0,0)%
   \put(0.5,-\value{chapter}){%
      {\color{black}\rule[-.2\unitlength]{2\unitlength}{\unitlength}}}%
   \put(0.8,-\value{chapter}){\makebox(0,.5)[l]{%
      {\color{white}\large\sffamily\thechapter}}}%
 \end{picture}\else\relax\fi}%
%
\def\ps@myhead{%
  \def\@evenfoot{% Even Foot
     \if@mparswitch\hss\fi%
        \hbox to \fullwidth{%
           \autoxspacing\textbf{\thepage}\hfill}%
    \if@mparswitch\else\hss\fi}%
  \def\@oddfoot{% Odd Foot
        \hbox to \fullwidth{%
           \autoxspacing\hfill\textbf{\thepage}}\hss}%
  \def\@evenhead{% Even Head
    \if@mparswitch \hss \fi
    \underline{\hbox to \fullwidth{\autoxspacing
        \leftmark\hfill\@title\draft@mark}}%
    \if@mparswitch\else \hss \fi}%
  \def\@oddhead{%
     \underline{%
        \hbox to \fullwidth{%
          \autoxspacing\@title\hfill{\if@twoside\rightmark\else\leftmark\fi}%
  }}{\rtume\draft@mark}\hss}%
  \let\@mkboth\markboth
  \def\chaptermark##1{\markboth{% chaptermark
    \ifnum \c@secnumdepth >\m@ne
      \if@mainmatter
        \@chapapp\thechapter\@chappos\hskip1zw
      \fi
    \fi
    ##1}{}}%
  \def\sectionmark##1{\markright{% sectionmark
    \ifnum \c@secnumdepth >\z@ \thesection \hskip1zw\fi
    ##1}}%
}% end of ps@myhead
\def\ps@plainhead{%
  \let\@mkboth\@gobbletwo
  \def\@oddhead{\hbox to\fullwidth{\autoxspacing\hfill\rtume\draft@mark}\hss}%
  \let\@evenhead\@empty
  \def\@evenfoot{%
    \if@mparswitch \hss \fi
    \hbox to \fullwidth{\textbf{\thepage}\hfil}%
    \if@mparswitch\else \hss \fi}%
  \def\@oddfoot{%
    \hbox to \fullwidth{\hfil\textbf{\thepage}}\hss}}
%
\let\ps@plain\ps@plainhead
%    \end{macrocode}
%
% \DescribeMacro{\pagestyle}
% ページスタイルを指定します。
%    \begin{macrocode}
\pagestyle{myhead}
%    \end{macrocode}
%
%
% \subsection{表紙ページのスタイルの変更}
% 
% \DescribeMacro{\maketitle}
% 表題ページは章見出しと統一します。
%    \begin{macrocode}
\renewcommand{\maketitle}{%
  \begin{titlepage}%
    \parindent=0pt%
    \let\footnotesize\small%
    \let\footnoterule\relax%
    \let\footnote\thanks%
    \null\vskip3zw%
    \hbox to \fullwidth{%
        \huge\bfseries\@title\hfill}\par\vskip\fboxsep
    \hrule height 1ex\par\vskip\fboxsep
    \hbox to \fullwidth{\hfill{\large\@author}}\par
    \hbox to \fullwidth{\hfill{\large 第{\jFileVersion}版}}\par
    \hbox to \fullwidth{\hfill{\large\@date}}\par
    \vfill
    \ifx\@contact\@empty\@thanks\vfil\null%
    \else\vfill\@contact\vskip 3ex\fi%
  \end{titlepage}%
\setcounter{footnote}{0}%
\thispagestyle{empty}
\vspace*{\fill}
{\hfil Copyright {\copyright} 2005, 2006 渡辺徹\hfil}\\
\begin{quotation}
この文書をフリーソフトウェア財団発行のGNUフリー文書
利用許諾契約書 (バージョン1.1かそれ以降から一つを選択) 
が定める条件の下で複製,頒布,あるいは改変することを
許可する.変更不可部分,表カバーテキスト,裏カバーテキ
ストは指定しない.この利用許諾契約書の複製物は
\emph{GNU Free Documentation License}\pp{GNUフリー文書
利用許諾契約書}という章\pp{付録~\ref{label_fdl}}に
含まれている.
\end{quotation}
\begin{quotation}
本冊子に記載されている企業,団体の名前や製品名等は
それぞれの権利帰属者の商標または商標登録であり所有物です.
本冊子では {\texttrademark} 及び {\textregistered} は明記し
ていません.
\end{quotation}%
}
%    \end{macrocode}
%
% \subsection{奥付や口絵}
% 
% \DescribeMacro{\printokuduke}
% 奥付を出力するための |\printokuduke| コマンドを用意します。
%     \begin{macrocode}
\newcommand{\printokuduke}{{%
  \if@twocolumn \onecolumn \else \clearpage \fi
  \thispagestyle{empty}%
  \null\vfill
  \parindent=0pt
  {\LARGE\bfseries\@title\hfill}\par\vskip1zw
  {\hfill\copyright\space\@author\space 2005, 2006}\par\vskip1zw
  {\hrule width\textwidth height1.5pt}\par\vskip1zw
  \begin{tabular}{ll}
    発行日 & 2005年 05 月  29 日 第 0.1 版 配布\\
           & 2006年 02 月  ?? 日 第 0.1a 版 配布\\
%	   & {\@date}  第{\@FileVerjou}版 配布\\
    編集   & {\@author}                     \\
  \end{tabular}\hfill
  {\hrule width\textwidth height1.5pt}%
  }%
}
%    \end{macrocode}
%
% \DescribeMacro{\kuchie}
% 口絵ページを出力するための命令です。
%
%    \begin{macrocode}
\newcommand\kuchie{%
  \if@openright
    \cleardoublepage
  \else
    \clearpage
  \fi
  \@mainmatterfalse
  \pagenumbering{Roman}}
%    \end{macrocode}
%
% 
% \subsection{その他の定義}
% 
% URL の定義などをします。
%     \begin{macrocode}
%\urldef{\webFDL}{\url}{http://www.gnu.org/copyleft/fdl.ja.html}
%\urldef{\webDante}{\url}{http://tex.dante.jp/}
%\urldef{\webOkumura}{\url}{http://oku.edu.mie-u.ac.jp/~okumura/texwiki/}
%\urldef{\webJabref}{\url}{http://jabref.sourceforge.net/}
%\urldef{\webDvipdfmx}{\url}{http://project.ktug.or.kr/dvipdfmx/}
%\urldef{\webGoogle}{\url}{http://www.google.co.jp} 
%\urldef{\webYatex}{\url}{http://www.yatex.org/}
%
\def\cleardoublepage{%
  \clearpage
  \if@twoside
    \ifodd \c@page\else
      \null\vfill
      \thispagestyle{empty}%
      \begin{flushright}%
         \includegraphics[scale=.4]{images/gnu-head}%
      \end{flushright}%
      \clearpage
    \fi
  \fi
}
%    \end{macrocode}
%
%    \begin{macrocode}
%</lll>
\endinput
%    \end{macrocode}
%
% \Finale

