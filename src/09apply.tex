%#!platex tls.tex
\chapter{\TeX とその周辺\zdash 応用}

\begin{abstract}
いつも何気なく呪文のように呼び出している\LaTeX コマンドたち.
この章では,\LaTeX 周辺の日陰で過ごすプログラムたちに焦点を
当ててみようと思います.
\end{abstract}

\section{基本的なコマンドの詳細}
普段は何気なく \type{platex filename.tex} 等としてタイプセットしている
かも知れませんが,\LaTeX に関わる基本的なコマンドも,他の Unix ツールと
同じように\Z{コマンドラインオプション}を持っています.

ここでは\LaTeX に関わる基本的なコマンドのオプションとその使い方などを紹
介します.

\subsection{\TeX の正当なインストール方法}
現在,Windows,Macintosh,Linux などの多くの OS において,\TeX の
導入は簡単になりました.しかし,\TeX の\Z{インストール}方法や\Z{動作原理}を
知らないと,思いがけない箇所でつまずく事になりかねません.
\TeX がどのようにインストールされるのかを知る事により,それらに
関わる問題を発見しやすくなります.

まず,\TeX は \Prog{Pascal} に似た言語 \Prog{WEB} で書かれています.
要するにKnuthが\TeX を開発しやすいようにPascalに修正を加えた言語です.

\begin{append}
 この節はまだ執筆途中.
\end{append}

%&ref(http://www.ring.gr.jp/pub/text/CTAN/systems/web2c/web.tar.gz);
%&ref(http://www.ring.gr.jp/pub/text/CTAN/systems/web2c/web2c.tar.gz);

%tex.web --(weave)--> tex.tex --(tex/pdftex)--> tex.dvi/tex.pdf
%tex.web --(tangle)--> tex.p + tex.pool

%Pascal はちょっと,という型のために ctangle/cweave というファイル名
%になります.

%tex.web --(weave)--> 
%tex.web tex.ch --(tangle)--> tex.p + tex.pool
%tie: marge or apply WEB change files

しかし,実際には移植性等の種々の問題に対応するために現在は WEB2C という
機構を使って WEB から C のソースを生成するようになっています.

\subsection{インストール}

\TeX をインストールする際に,
歴史的には \Prog{initex} と \Prog{vartex} の2つが提供されていましたが,
今は \TeX にその両方の機能が実装されています.

\begin{InTerm}
 \type{tex -ini latex.ini \\dump}
 \type{cp tex latex}
\end{InTerm}
近年であれば上記のようにコマンドを打てばインストールは出来るでしょう.

専ら te\TeX に含まれる \prog{fmtutil}を使って
\begin{InTerm}
 \type{fmtutil --all}
\end{InTerm}
とするのが一番簡単でしょう.

\XeTeX をインストールするというような状況であれば
\begin{InTerm}
 \type{xeinitex xetex.ini \\dump}
\end{InTerm}
とすれば \fl{xetex.fmt}, \fl{xetex.pool}の二つが作成されるので,
これを適当な場所,例えば\dir{/usr/local/share/texmf/web2c/}に
コピーします.

%&ref(http://macptex.appi.keio.ac.jp/~uchiyama/macosx/)

initex は現在 tex -ini のシンボリックリングで,
vartex は現在 tex のシンボリックリンクであるからして,
\begin{InTerm}
 \type{ptex -ini -kanji=sjis -jobname=platex-sjis platex.ini}
 \type{ptex -fmt=platex-sjis -progname=platex -kanji=sjis filename}
\end{InTerm}
\fl{platex-sjis.fmt}というファイルを作成しておけば,
EUC で make された \pTeX も SJIS の \fl{filename.tex} を
タイプセットできるようになるでしょう.


%\begin{quote}
% \Fl{bibitex.web}, \Fl{mf.web}, \Fl{mp.web}, \Fl{tangle.web}, \Fl{tex.web}, 
% \Fl{weave.web}. 
%\end{quote}

%tex.web -> tex.p, tex.tex -> tex.pdf
%
%initex virtex (symbolic link) tex --ini 
%--fmt
%
%tex.pool
%
%plain.base
%plain.fmt
%plain.mem
%
%platex-euc.fmt
%platex-jis.fmt
%platex-sjis.fmt
%
%platex.fmt -> argv[0]

%texmf.cnf ももちろん参照されるが,これは
%
%tex.fmt ファイルの生成
%
%.mf -> tfm, pk, gf, pfb, 
%
%mktexftm  -> fmtutil: utility for maintaining TeX format files
%mktexlsr
%mktexmf
%mktexpk
%mktextfm
%
%fmtutil --byfmt formatname
%fmtutil --byhyphen hyphenfile (teTeX)
%fmtutil.cnf configuration file for fmtutil (teTeX)



\subsection{\TeX/\pTeX}
\TeX を実行するにはコンソールから \type{tex} とすれば,次のように
表示されます.
\begin{OutTerm}
This is TeX, Version 3.14159 (Web2C 7.4.5)\\
**
\end{OutTerm}
これはユーザとの\Z{インタラクティブ}な操作を許容している状態です.
ここで何らかのファイル名 \Va{filename}{tex} の \va{filename} を
指定すると,ファイルが存在する場合は,それがタイプセットされます.
存在しない場合は次のようにエラーを吐きます.
\begin{OutTerm}
! I can't find file `<filename>'.\\
<*> <filename>\\
Please type another input file name: 
\end{OutTerm}
とりあえず終了したければ `\str{*}' が表示されている状態で \C{bye} と
打ち込む事になります.\TeX は実行時の引数としてタイプセットする原稿の
ファイル名を指定します.
\begin{Syntax}
 \str{tex} \opa{オプション} \opa{ファイル名[.tex]}
\end{Syntax}
原稿ファイル名の拡張子である \suf{tex} は省略できます.

\begin{description}
 \item[\copt{-file-line-error-style}]
    print file:line:error style messages
 \item[\copt{-interaction=STRING}]      
    set interaction mode (STRING = batchmode, nonstopmode, scrollmode,
	    errorstopmode)
 \item[\copt{-jobname=STRING}]          set the job name to STRING
 \item[\copt{-parse-first-line}]        parse of the first line of the input file
 \item[\copt{-progname=STRING}]         set program (and fmt) name to STRING
 ですから,例えば \type{ptex -progname=platex} とすれば,\str{platex} を
 実行したのと同じ効果が得られる事になります.
 \item[\copt{-recorder}]                enable filename recorder
 \item[\copt{-shell-escape}]            enable \write18{SHELL COMMAND}
 \item[\copt{-src-specials}]            insert source specials into the DVI file
 \item[\copt{-src-specials=WHERE}]      insert source specials in certain places of
                          the DVI file. WHERE is a comma-separated value
                          list: cr display hbox math par parend vbox
 \item[\copt{-translate-file=TCXNAME}]  use the TCX file TCXNAME
 \item[\copt{-help}]                    display this help and exit
 \item[\copt{-version}]                 output version information and exit
\end{description}



\subsection{\LaTeX/\pLaTeX}



\subsection{xdvi/dviout}



\section{Ghostscript}

\subsection{pdf2ps}

\GS 付属のツールで PDF ファイルを \PS ファイルに変換する.変換精度はあま
り満足の行くものではないことが多い.

\prog{pdf2ps} を使うよりは \Xpdf utilities の \prog{pdftops} を使う方が
良い.


\subsection{ps2pdf}

\GS 付属の \PS を PDF に変換するためのプログラム.

GNU \GS 7.07 などを使っており,和文書体を用いている場合は
\begin{InTerm}
\type{ps2pdf14 -dNOKANJI file.ps file.pdf}
\end{InTerm}
などとすると \Z{Ryumin-Light} などの(和文)書体を埋め込まないように処理する.

AFPL Ghostscript 8.50 では TrueType 系のフォントを埋め込むことができるよ
うになった (今までバグのために出来なかった) ので
\begin{InTerm}
 \type{ps2pdf14 file.ps file.pdf}
\end{InTerm}
としても良い.

Mac~OS~Xにはpstopdf というAppleオリジナルのPS-PDF 変換のためのコマンド
がある.

Vine~Linux では \prog{ps2jpdf} という便利なコマンドもある.


\section{TDS}

\TeX\ directory structure

\section{Kpathsearch}

kpsewhich 

\section{texdoc}



\section{WEB2C}

WEB や tangle/weave

\subsection{WEB}

`Web' と表記しない限り,\TeX の世界において WEB は Knuth が開発した「プログラムのソースコードと説明文書の混在形式を実現する文書整形言語」となります\footnote{Knuth が計算機科学領域で使用されていない 3 文字からなる英単語を参考に Knuth の義母(妻 Jill の母親) Wilda Ernestine Bates のイニシャルから命名したものです.}.

WEB には Pascal 風(若干の訛りと方言がある) の「ソースコード」と フォー
マットの決められた「説明文」が埋め込まれています.例としては次のようなも
のがあります.
\begin{Pascaly}
% This line is a comment.
@* Introduction.
@p function round_decimals(@!k:small_number) : scaled; {converts a decimal fraction}
var $a$:integer; {the accumulator}
begin $a$ := $0$;
  while $k > 0$ do
    begin decr($k$); $a$ := ($a$ + dig[$k$]*two) div $10$; 
    end;
  round_decimals := ($a + 1$) div $2$; 
end;
@* Indexes.
\end{Pascaly}
生徒が begin と end の使い方を良く間違えると,Knuth 先生は嘆いておられま
した.このようにして記述された WEB ファイルは weave と tangle によって処
理できます.

現在は WEB ではなく C 言語で記述する CWEB が主流になっているそうです.
CWEB での手順を紹介します.次のようにすると \suf{web} ファイルから マニュ
アル\fl{test.tex} が生成されます.
\begin{InTerm}
 \type{cweave test.w}
\end{InTerm}
次のようにすると \suf{web} ファイルから C ソースコード \fl{test.c} が生
成されます.
\begin{InTerm}
 \type{ctangle test.w}
\end{InTerm}
ですから, \fl{Makefile}においての依存関係は次のようになります.
\begin{Makefile}
.w.tex:
	cweave $*
.tex.dvi:	
	tex $<
.w.dvi:
	make $*.tex
	make $*.dvi
.w.c:
	ctangle $*
.w.o:
	make $*.c
	make $*.o
.c.o: 
	cc $(CFLAGS) -c $*.c
.w:
	make $*.c
	cc $(CFLAGS) $*.c -o $*
\end{Makefile}
`\str{.tex.dvi}' に関しては
\begin{Makefile}
.tex.pdf:
	pdftex $< && pdftex $<
\end{Makefile}
としてしまって PDF を作成するようにしても良いでしょう.

詳しい情報は CTAN の \dir{web/c_cpp/cweb/examples/} を参照する事で
相当な \TeX nician になれる事でしょう.


\subsection{weave}

weave とは WEB で記述された「ソースコード+説明」混在文書からソースコー
ドのマニュアルを作成するツールです.weave にはプログラムを明瞭な文書に織
り上げる (weave) という意味があります.

例えば,次のようなファイル \fl{test.web} が存在するとします.
\begin{Pascaly}
% This line is a comment.
@* Introduction.
@p function round_decimals(@!k:small_number) : scaled; {converts a decimal fraction}
var $a$:integer; {the accumulator}
begin $a$ := $0$;
  while $k > 0$ do
    begin decr($k$); $a$ := ($a$ + dig[$k$]*two) div $10$; 
    end;
  round_decimals := ($a + 1$) div $2$; 
end;
@* Indexes.
\end{Pascaly}
パーセント `\%' で記述される行はコメントとして扱われます.\str{@*} は節
見出しの始まりで,ドットが出現するまで見出し語として扱われます. このよ
うなファイルを
\begin{InTerm}
 \type{weave test.web}
\end{InTerm}
とすると test.tex が作成されます.この生成されたマニュアルを
\begin{InTerm}
 \type{pdftex test.tex}
 \type{pdftex test.tex}
\end{InTerm}
として処理すると PDF ファイル \fl{test.pdf} が生成されます.面白いので,ご自
身で処理してみてください(もちろん,目次・索引の自動生成が行なわれます).


\subsection{tangle}

\prog{tangle} とは \Z{WEB} で記述された「ソースコード+説明」文書から,ソースコード部分を手繰り寄せるツールです.

例えば,次のような \fl{test.web} が存在するとします.
\begin{Pascaly}
% This line is a comment.
@* Introduction.
@p function round_decimals(@!k:small_number) : scaled; {converts a decimal fraction}
var $a$:integer; {the accumulator}
begin $a$ := $0$;
  while $k > 0$ do
    begin decr($k$); $a$ := ($a$ + dig[$k$]*two) div $10$; 
    end;
  round_decimals := ($a + 1$) div $2$; 
end;
@* Indexes.
\end{Pascaly}

これを tangle プログラムで
\begin{InTerm}
 \type{tangle test.web}
\end{InTerm}
とすると 次のような \fl{test.p} が作成されます.
\begin{InText}
{1:}function rounddecimals(k:smallnumber):scaled;var a:integer;
begin a:=0;while k>0 do begin decr(k);a:=(a+dig[k]*two)div 10;end;
rounddecimals:=(a+1)div 2;end;{:1}
\end{InText}
インデントなどがされていないので,読みづらいですが,機械が必要とするソー
スコードだけが抜き出されます.



\subsection{weave}



\section{PDF 操作}

Dvipdfmx, Xpdf utilities, 

\section{\texorpdfstring \eTeX {eTeX}}
Knuth が開発した \TeX を拡張したもの.Knuth の時代の計算機環境と今の計算
機環境は結構違うので,\TeX に対して贅沢にリソースを分けることができるので,
レジスタ数なども増えている.プリミティブの追加も行なわれているはず.


\section{\texorpdfstring \pdfTeX {pdfTeX}}


\section{\texorpdfstring \XeTeX {XeTeX}}


\section{多言語組版への道}

\subsection{Aleph}

\TeX で多言語組版を実現する一つの方法として Omega を使うことが考えられた
が \TeX をベースにするよりも \eTeX をベースにした方が良いだろうという事に
なった. \eTeX をベースにユニコードベースで多言語組版を実現する plain \TeX
用のプログラムが Aleph であり, \LaTeX 用は Lamed と呼ばれる.

\section{Lambda}

\TeX で眞の多言語組版を実現するのはちょっと難しい.組み方向の違い,行送
りの違い,ハイフネーションの違い,放言の違い,スペルの違い,字並びの違い,
スペーシングの違い等など例をあげればきりがない.例えば, LR, RL, TD の組
み方向が混在する文書をやるのは相当手続きが多い. unicode をベースにユー
ザーが入力メソッドを意識することなく,あらゆる言語をあらゆる放言つきで組
版出来れば便利だ.ということで Omega (plain \TeX 用) が開発された.もち
ろん \LaTeX 用が Lambda というわけだ.しかし,現在開発は\ldots \eLaTeX
をマージした Lamed が続いている模様.

\section{Lamed}

\TeX で多言語組版(表意文字圏を含む)を実現する一つの方法として Omega や
Lambda を用いることも考えられるが,\eTeX をベースにした方が良いだろうとい
う流れになってきた.\eTeX をベースに unicode による多言語組版を実現するプ
ログラムが Lamed である,これは \LaTeX 用で, \TeX 用は Aleph という名前で
ある.



\section{周辺プログラム}

\subsection{mkTeXfmt}

fmtutils の一種で \TeX の フォーマットファイルを作成するために使われる.

\subsection{mkTeXtfm}

\MF のソース \Va{file}{mf} から TeX プログラムがフォントを扱う上で必要に
なるメトリクス情報を保存した \Va{file}{tfm} を作成するためのプログラム.
\begin{InTerm}
\type{mktextfm --destdir /tmp/ cmr10 }
\end{InTerm}
などとすれば \dir{/tmp} に\fl{cmr10.tfm} なるファイルが作成される.

日本語の等の場合は JFM というファイルが必要になったり, Omega/Lambda な
どの場合は異なる等の話もある.


\subsection{mkTeXpk}

\MF ソースから PK フォントを生成するためのプログラム.dviware などは \MF
のソース \Va{file}{mf} を必要とするわけではなくビットマップのフォントデー
タが格納された PK フォントを必要とすることになる.
\begin{InTerm}
 \type{mktexpk --destdir /tmp/ cmr10}
\end{InTerm}
などとすると \dir{/tmp} に \fl{cmr10.600pk} なるファイルが作成される.通
常は解像度 (dpi) を指定することになり,標準では 600\,dpi となり,ファイル
名が \fl{cmr10.600pk} のように拡張子が `解像度pk' になっている.標準的な
出力ディレクトリは \dir{$texmf/fonts/pk} など.昨今の dviware は PK フォ
ントではなく,解像度に依存しない TrueType フォントや PostScript Type 1,
OpenType フォント等の表示に対応しているため,少々 obsolete なフォント形
式と言える.

\subsection{mktexlsr}

\TeX に関わるファイル郡が格納されているディレクトリ
\dir{/usr/share/texmf/} などで
\begin{InTerm}
 \type{ls -R /usr/share/texmf >ls-R}
\end{InTerm}
としたもの.\TeX に関わるファイルは年々増大し,プログラム自身もどこにど
のファイルがあるのかを検索するのが面倒になってきた.そのため,あらかじめ
どこにどのファイルがあるのかを記述したファイル \Fl{ls-R} が必要になって
きた.その後 \type{ls -R} の出力結果が環境により異なるという事情から
\prog{mktexlsr} という専用のプログラムが配布されるようになった.新たに
\TeX が格納されているディレクトリにファイルを追加した場合,Linux 系 OS
であれば
\begin{InTerm}
\type{mktexlsr}
\end{InTerm}
とする.しかし,近年はハードディスク等も高速になっていることから,
\fl{ls-R}を必要としないので
\begin{InTerm}
 \type{rm `kpsewhich --expand-path='$TEXMFMAIN'`/ls-R}
\end{InTerm} 
として \fl{ls-R} を削除してしまっても構わない.むしろ,削除した方がファ
イルの追加を気がねなく行なえるかもしれない.

\TeX のファイルが格納されているディレクトリは \emph{TDS: TeX Directory
Structures}という指針に乗っ取り,ディレクトリ構造を決めることが望ましい
事になっている.

\section{\texorpdfstring \ConTeXt {ConTeXt}}

\subsection{\TeX Exec}

\ConTeXt 付属のタイプセット支援スクリプト.
\begin{InTerm}
 \type{texexec --pdfarrange --result=output.pdf file1.pdf file2.pdf file3.pdf}
\end{InTerm}
などとすると \fl{file1.pdf}, \fl{file2.pdf}, \fl{file3.pdf} 
を一つの PDF \fl{output.pdf} に編集できる.

\section{\TeX info}

GNU project で推奨しているマニュアル作成または文書作成用の 
plain TeX をベースにしたシステム.例えば,次のようなファイル 
\fl{file.texi} があったとする(拡張子は \suf{txi}とか \suf{texi}とか
いろいろある).
\begin{InText}
 \def\lang{jp}
 \input texinfo
 @settitle Hello, Texinfo!
 \bye
\end{InText}
これを
\begin{InTerm}
 \type{ptex file.texi}
 \type{texindex file.??}
 \type{ptex file.texi}
 \type{dvipdfmx file.texi}
\end{InTerm}
などとすることにより PDF に変換できる.これ以外に
 texi2dvi, texi2pdf, texi2html などが用意されている.


\subsection{texindex}

Texinfo のファイル \fl{file.txi} を
\begin{InTerm}
 \type{tex file.txi}
\end{InTerm}
としても索引などは生成されないため,別途 索引を生成するために texindex
プログラムを実行する必要が或.
\begin{InTerm}
\type{texindex file.??}
\end{InTerm}
そうして,もう一度
\begin{InTerm}
 \type{tex file.txi}
\end{InTerm}
とすることで \fl{file.dvi} を作成できる.

