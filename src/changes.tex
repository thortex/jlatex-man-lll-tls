%
\RequirePackage{csquotes}\relax
%
\makeatletter
% 連絡先変わりました.
\contact{%
   \parbox{.9\textwidth}{Th\'or Watanabe\\%
   Graduate School of System Information Science\\%
   Future University-Hakodate\\%
   thor@tex.dante.jp\\%
   \url{http://tex.dante.jp/}}}%
%
% 人名
\def\person@shi{氏}
\def\hito#1#2{#1#2\person@shi}
\def\Hito#1#2{%
  \index{人名!#1#2}%
  \index{#1#2}#1#2\person@shi}
\def\Person#1#2{#1 #2}
\def\person#1#2{%
  \index{人名!#2, #1}%
  \index{#2, #1}#1 #2\person@shi}
%
% 相互参照
\def\chaplab#1{\label{chap:#1}}%     章のラベル
\def\chapref#1{第~\ref{chap:#1}~章}% 章の参照
\def\seclab#1{\label{sec:#1}}%       節のラベル
\def\secref#1{\ref{sec:#1}~節}%      節の参照
\def\figlab#1{\label{fig:#1}}%       図のラベル
\def\figref#1{図~\ref{fig:#1}}%      図の参照
\def\tablab#1{\label{tab:#1}}%       表のラベル
\def\tabref#1{表~\ref{tab:#1}}%     表の参照
\def\equlab#1{\label{equ:#1}}%       式のラベル
\def\equref#1{式~\ref{equ:#1}}%     式の参照
\def\applab#1{\label{app:#1}}%       付録のラベル
\def\appref#1{付録~\ref{app:#1}}%    付録の参照
\def\pref#1{\pageref{#1}ページ}%     ページの参照
\def\exelab#1{\label{exe:#1}}%       例題ラベル
\def\exeref#1{例題~\ref{exe:#1}}%    例題参照
\def\problab#1{\label{prob:#1}}%     問題ラベル
\def\probref#1{問題~\ref{prob:#1}}%  問題参照

% 略称
\def\gnu      {\textsc{Gnu}\xspace}
\def\gpl      {\gnu General Public License\xspace}
\def\fdl      {\gnu Free Documentation License\xspace}
\def\jfdl     {\gnu フリー文書利用許諾契約書}
\def\thefdl   {The \gnu Free Documentation License}
\urldef \webCopyLeft    \url {http://www.gnu.org/copyleft/}
\urldef \webGNU         \url {http://www.gnu.org/}
\newcommand*\ascii          {Ascii\xspace}
%
% プログラム名
\DeclareRobustCommand*\TeXforHT{\TeX4HT\xspace}
\DeclareRobustCommand*\LaTeXtoHTML{\LaTeX2HTML\xspace}
\DeclareRobustCommand*\dviout{Dviout\xspace}
\DeclareRobustCommand*\ConTeXt{Con\TeX t\xspace}
\DeclareRobustCommand*\TtH{T\hskip.1em\lower.3ex\hbox{T}H\xspace}
\DeclareRobustCommand*\Xpdf{Xpdf\xspace}
%
% ディレクトリ・フォルダの表記
\def\dir{\begingroup \urlstyle{tt}\Url}
%
% 一般的な code の設定
\lstset{%
  language={C},%
  backgroundcolor={},%
  basicstyle={\small\ttfamily},%
  identifierstyle={\small\ttfamily},%
  commentstyle={\small\normalfont},%
  keywordstyle={\small\ttfamily\bfseries},%
  ndkeywordstyle={\small\ttfamily},%
  stringstyle={\small\sffamily},
  frame={tb},
  breaklines=true,
  columns=fullflexible,%
  numbers=left,%
  stepnumber=5,%
  xrightmargin=0zw,%
  xleftmargin=1.5zw,%
  numberstyle={\scriptsize},%
  numbersep=1zw,%
  lineskip=-0.5ex%
}
%
% GNU Make 用
\lstnewenvironment{Makefile}{%
  \lstset{language=Make,%
    showtabs,tab=\rightarrowfill,%
    stepnumber=0,%
    basicstyle={\small\ttfamily}}}{}%
%
% Pascal 風のコード用
\lstnewenvironment{Pascaly}{%
  \lstset{language=Pascal,%
    showtabs,tab=\rightarrowfill,%
    stepnumber=0,%
    mathescape,%
    keywordstyle={\small\bfseries},%
    basicstyle={\small\rmfamily}}}{}%
%
\def\@sect#1#2#3#4#5#6[#7]#8{%
  \ifnum #2>\c@secnumdepth
    \let\@svsec\@empty
  \else
    \refstepcounter{#1}%
    \protected@edef\@svsec{\@seccntformat{#1}\relax}%
  \fi
  \@tempskipa #5\relax
  \ifdim \@tempskipa<\z@
    \def\@svsechd{%
      #6{\hskip #3\relax
      \@svsec #8}%
      \csname #1mark\endcsname{#7}%
      \addcontentsline{toc}{#1}{%
        \ifnum #2>\c@secnumdepth \else
          \protect\numberline{\csname the#1\endcsname}%
        \fi
        #7}}% 目次にフルネームを載せるなら #8
  \else
    \begingroup
      \interlinepenalty \@M % 下から移動
%%%%%%
    \ifnum #2=\@ne
      \colorbox[cmyk]{0,0,0,\my@default@gray}{%
        \parbox\ftextwidth{#6{\@hangfrom{\@svsec}#8}%
          \hfil}%
        }\@@par%
    \else
        \ifnum #2=\tw@
           {\large$\blacktriangledown$\hskip3pt}%
           #6{\@hangfrom{\hskip #3\relax\@svsec}%
              \relax#8\@@par}%
        \else
           #6{\@hangfrom{\hskip #3\relax\@svsec}#8\@@par}%
        \fi
    \fi
    \endgroup
%%%%%
    \csname #1mark\endcsname{#7}%
    \addcontentsline{toc}{#1}{%
      \ifnum #2>\c@secnumdepth \else
        \protect\numberline{\csname the#1\endcsname}%
      \fi
      #7}% 目次にフルネームを載せるならここは #8
  \fi
  \@xsect{#5}}
%
\renewcommand{\printokuduke}{{%
  \if@twocolumn \onecolumn \else \clearpage \fi
  \thispagestyle{empty}%
  \null\vfill
  \parindent=0pt
  {\LARGE\bfseries\@title\hfill}\par\vskip1zw
  {\hfill\copyright\space\@author\space 2005, 2006}\par\vskip1zw
  {\hrule width\textwidth height1.5pt}\par\vskip1zw
  \begin{tabular}{ll}
    発行日 & 2005年 05 月  29 日 第 0.01 版 配布\\
           & 2006年 08 月  31 日 第 0.10 版 配布\\
    編集   & {\@author}                     \\
  \end{tabular}\hfill
  {\hrule width\textwidth height1.5pt}%
  }%
}
%
\AtBeginDocument{\def\Gin@extensions{.pdf,.bmp,.png,.jpg,.jpeg,.eps,.ps}}
\makeatother
%