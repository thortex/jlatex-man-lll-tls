\chapter{更新履歴}

%2006 年 8 月 xx 日にウェブ上でこの冊子の ver.~0.1 を公開しました.
%誤植等が\kenten{たくさん}あると思います.

%\section*{あれっおかしいな?と思ったら}
この文書は私一人で執筆しておりますから,
どこかに間違いや誤植がある確率が高くなっています.
「あれっおかしいな?」と思う箇所がありましたら
私のホームページ\footnote{\url{http://tex.dante.jp/}}%
の掲示板かメールアドレス\footnote{thor(at)tex(dot)dante(dot)jp}%
にご連絡ください.

% 変更履歴専用の list 型環境 changelog
\newenvironment{changelog}
 {\list{}{%
     \setlength{\leftmargin}{3zw}%
     \setlength{\labelwidth}{0pt}%
     \setlength{\labelsep}{0pt}%
     \setlength{\itemsep}{0pt}%
     \setlength{\parsep}{0pt}%
     \let \makelabel = \releaselabel}}
  {\endlist\ignorespacesafterend}

\makeatletter
\newcommand*{\releaselabel}[1]{%
  \if @#1@%
    \global\@itempenalty\z@
  \else
    \global\@itempenalty\@M
    \hspace{-\leftmargin}\sffamily
    \@releaselabel #1@%
  \fi}
\def\@releaselabel#1 #2@{%
  \makebox[\leftmargin-0.5em][r]{#1}%
  \hspace{0.5em}#2}
\makeatother

\begin{comment}
変更履歴の執筆にあたっては,1--5 を区別して記述すること.
 \begin{itemize} 
  \item[削除 (deleted)]   ある要素を取り除く事.
  \item[追加 (added)]     ある要素を新規に取り入れる事.
  \item[加筆 (extended)]  すでにある要素の小要素を追加する事.
  \item[修正 (revised)]   すでにある要素に変更を加える事.
  \item[改訂 (updated)]   文書を改めて訂する事.
  \item[誤植 (fixed bug)] 誤った記述.訂正されるべき要素.
 \end{itemize}
\end{comment}

\begin{changelog}
 \item[0.10 2006/08/31]
   \item 初版を公開しました.
 \item[0.01 2005/05/29] 
   \item テスト版を公開しました.
\end{changelog}
